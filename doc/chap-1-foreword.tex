%--------------------------------------------------------------------
% Foreword: audience, authors, license, installation
%--------------------------------------------------------------------
\chapter{Foreword}
\label{c:1:foreword}

\section{Audience}
This document is for users of \safe (Scalable Analysis Framework for ECMAScript)
2.0, a scalable and pluggable analysis framework for JavaScript web applications.
General information on the \safe project is available at an invited talk at ICFP 2016~\cite{safeicfp16}:
\begin{center}
  \url{https://www.youtube.com/watch?v=gEU9utf0sxE}
\end{center}
and the source code and publications are available at:
\begin{center}
  \url{https://github.com/sukyoung/safe}
\end{center}
For more information, please contact the main developers of \safe
at \texttt{safe [at] plrg.kaist.ac.kr}.

SAFE has been used by:
\begin{itemize}
\itemsep-.2em
\item JSAI~\cite{jsai} @ UCSB
\item ROSAEC~\cite{rosaec} @ Seoul National University
\item K framework~\cite{kjs} @ UIUC
\item Ken Cheung~\cite{emse16} @ HKUST
\item Web-based vulnerability detection~\cite{oracle} @ Oracle
\item Tizen~\cite{tizen} @ Linux Foundation
\end{itemize}

\section{Contributors}
The main developers of SAFE 2.0 are as follows:
\begin{itemize}
\itemsep-.2em
\item Jihyeok Park
\item Yeonhee Ryou
\item Sukyoung Ryu
\end{itemize}
and the following have contributed to the source code:
\begin{itemize}
\itemsep-.1em
\item Minsoo Kim (Built-in function modeling)
\item PLRG @ KAIST and our colleagues in S-Core and Samsung Electronics (SAFE 1.0)
\end{itemize}


\section{License}
The \safe source code is released under the BSD license:
\begin{center}
\url{github.com/sukyoung/safe/blob/master/LICENSE}
\end{center}


\section{Installation}
We assume you are using an operating system with a Unix-style shell
(for example, Mac OS X, Linux, or Cygwin on Windows).
Assuming \verb!SAFE_HOME! points to the SAFE directory,
you will need to have access to the following:
\begin{itemize}
\item J2SDK 1.8.  See
\url{http://java.sun.com/javase/downloads/index.jsp}
\item Scala 2.12.  See
\url{http://scala-lang.org/download}
\item sbt version 0.13.  See
\url{http://www.scala-sbt.org}
\item Bash version 2.5, installed at \verb!/bin/bash!.  See
\url{http://www.gnu.org/software/bash/}
\end{itemize}

In your shell startup script, add \verb!$SAFE_HOME/bin! to your path.
The shell scripts in this directory are Bash scripts.
To run them, you must have Bash accessible in \verb!/bin/bash!.

Type \verb!sbt compile! and then \verb!sbt test! to make sure that
your installation successfully finishes the tests.
Two regression test suites are provided with \safe and can be
analyzed automatically:
\begin{verbatim}
$ sbt test
$ sbt test262Test
\end{verbatim}
In addition to the \safe-specific test suite,
\safe 2.0 has been tested using Test262, the official ECMAScript (ECMA-262) conformance suite:
\begin{center}
\url{https://github.com/tc39/test262}
\end{center}
Not a single test should end in a failure.

Once you have built the framework, you can call it from any directory,
on any JavaScript file, simply by typing one of available commands at a command line
as explained in \cref{3:refman}.

\subsection{IntelliJ configuration}
IntelliJ users can use IntelliJ 2016.2.4 with the latest Scala plugin as follows:
\begin{enumerate}
\item Create a new project from existing sources (aka. \verb!Import project!).
\item Choose \verb!build.sbt! in the SAFE 2.0 root to import.
\item Choose JDK 1.8 as the project JDK.
\item Manually download \verb!xtc.jar! in to \verb!lib/!
\item Goto \verb!Project Settings! $\rightarrow$ \verb!Modules! $\rightarrow$
\verb!root (module)! $\rightarrow$ \verb!Dependencies!
\item Open \verb!SBT:unmanaged-jars! dependencies.
\item Remove broken entries for \verb!spray-json! and \verb!xtc!.
\item Add \verb!(+) .jars! for the two libraries above.
\item Run the \verb!buildParsers! task in SBT.
\end{enumerate}

For debugging purpose, you can first add a configuration like this:

\begin{enumerate}
\item Go to \verb!Run -> Edit Configurations!
\item Click the ``+'' button on the left to add a new configuration, select ``Remote'' in the drop-down menu
\item Name the configuration something like ``remote-debugging'', then it should work out of box (you can check that the port is 5005 though)
\item Save the configuration by clicking ``OK''
\end{enumerate}

Now you can debug by first launching SAFE with a \verb!-debug! flag as the first argument, then following your other options. For example, \verb!./bin/safe -debug analyze something.js!. Your application is supposed to wait for connection from debugger once started, then you can start the debugging configuration we just created in IntelliJ.
