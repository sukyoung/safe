%%
%% This is file `sample-sigplan.tex',
%% generated with the docstrip utility.
%%
%% The original source files were:
%%
%% samples.dtx  (with options: `sigplan')
%% 
%% IMPORTANT NOTICE:
%% 
%% For the copyright see the source file.
%% 
%% Any modified versions of this file must be renamed
%% with new filenames distinct from sample-sigplan.tex.
%% 
%% For distribution of the original source see the terms
%% for copying and modification in the file samples.dtx.
%% 
%% This generated file may be distributed as long as the
%% original source files, as listed above, are part of the
%% same distribution. (The sources need not necessarily be
%% in the same archive or directory.)
%%
%% The first command in your LaTeX source must be the \documentclass command.
\documentclass[sigplan,10pt,review,anonymous]{acmart}\settopmatter{printfolios=true,printccs=false,printacmref=false}

%%
%% \BibTeX command to typeset BibTeX logo in the docs
\AtBeginDocument{%
  \providecommand\BibTeX{{%
    \normalfont B\kern-0.5em{\scshape i\kern-0.25em b}\kern-0.8em\TeX}}}

%% Rights management information.  This information is sent to you
%% when you complete the rights form.  These commands have SAMPLE
%% values in them; it is your responsibility as an author to replace
%% the commands and values with those provided to you when you
%% complete the rights form.
\setcopyright{acmcopyright}
\copyrightyear{2018}
\acmYear{2018}
\acmDOI{10.1145/1122445.1122456}

%% These commands are for a PROCEEDINGS abstract or paper.
\acmConference[Woodstock '18]{Woodstock '18: ACM Symposium on Neural
  Gaze Detection}{June 03--05, 2018}{Woodstock, NY}
\acmBooktitle{Woodstock '18: ACM Symposium on Neural Gaze Detection,
  June 03--05, 2018, Woodstock, NY}
\acmPrice{15.00}
\acmISBN{978-1-4503-XXXX-X/18/06}


%%
%% Submission ID.
%% Use this when submitting an article to a sponsored event. You'll
%% receive a unique submission ID from the organizers
%% of the event, and this ID should be used as the parameter to this command.
%%\acmSubmissionID{123-A56-BU3}

%%
%% The majority of ACM publications use numbered citations and
%% references.  The command \citestyle{authoryear} switches to the
%% "author year" style.
%%
%% If you are preparing content for an event
%% sponsored by ACM SIGGRAPH, you must use the "author year" style of
%% citations and references.
%% Uncommenting
%% the next command will enable that style.
%%\citestyle{acmauthoryear}

%%
%% end of the preamble, start of the body of the document source.
\begin{document}

%%
%% The "title" command has an optional parameter,
%% allowing the author to define a "short title" to be used in page headers.
\title{The Name of the Title is Hope}

%%
%% The "author" command and its associated commands are used to define
%% the authors and their affiliations.
%% Of note is the shared affiliation of the first two authors, and the
%% "authornote" and "authornotemark" commands
%% used to denote shared contribution to the research.
\author{Ben Trovato}
\authornote{Both authors contributed equally to this research.}
\email{trovato@corporation.com}
\orcid{1234-5678-9012}
\author{G.K.M. Tobin}
\authornotemark[1]
\email{webmaster@marysville-ohio.com}
\affiliation{%
  \institution{Institute for Clarity in Documentation}
  \streetaddress{P.O. Box 1212}
  \city{Dublin}
  \state{Ohio}
  \postcode{43017-6221}
}

\author{Lars Th{\o}rv{\"a}ld}
\affiliation{%
  \institution{The Th{\o}rv{\"a}ld Group}
  \streetaddress{1 Th{\o}rv{\"a}ld Circle}
  \city{Hekla}
  \country{Iceland}}
\email{larst@affiliation.org}

\author{Valerie B\'eranger}
\affiliation{%
  \institution{Inria Paris-Rocquencourt}
  \city{Rocquencourt}
  \country{France}
}

\author{Aparna Patel}
\affiliation{%
 \institution{Rajiv Gandhi University}
 \streetaddress{Rono-Hills}
 \city{Doimukh}
 \state{Arunachal Pradesh}
 \country{India}}

\author{Huifen Chan}
\affiliation{%
  \institution{Tsinghua University}
  \streetaddress{30 Shuangqing Rd}
  \city{Haidian Qu}
  \state{Beijing Shi}
  \country{China}}

\author{Charles Palmer}
\affiliation{%
  \institution{Palmer Research Laboratories}
  \streetaddress{8600 Datapoint Drive}
  \city{San Antonio}
  \state{Texas}
  \postcode{78229}}
\email{cpalmer@prl.com}

\author{John Smith}
\affiliation{\institution{The Th{\o}rv{\"a}ld Group}}
\email{jsmith@affiliation.org}

\author{Julius P. Kumquat}
\affiliation{\institution{The Kumquat Consortium}}
\email{jpkumquat@consortium.net}

%%
%% By default, the full list of authors will be used in the page
%% headers. Often, this list is too long, and will overlap
%% other information printed in the page headers. This command allows
%% the author to define a more concise list
%% of authors' names for this purpose.
\renewcommand{\shortauthors}{Trovato and Tobin, et al.}

%%
%% The abstract is a short summary of the work to be presented in the
%% article.
\begin{abstract}
  A clear and well-documented \LaTeX\ document is presented as an
  article formatted for publication by ACM in a conference proceedings
  or journal publication. Based on the ``acmart'' document class, this
  article presents and explains many of the common variations, as well
  as many of the formatting elements an author may use in the
  preparation of the documentation of their work.
\end{abstract}

%%
%% The code below is generated by the tool at http://dl.acm.org/ccs.cfm.
%% Please copy and paste the code instead of the example below.
%%
\begin{CCSXML}
<ccs2012>
 <concept>
  <concept_id>10010520.10010553.10010562</concept_id>
  <concept_desc>Computer systems organization~Embedded systems</concept_desc>
  <concept_significance>500</concept_significance>
 </concept>
 <concept>
  <concept_id>10010520.10010575.10010755</concept_id>
  <concept_desc>Computer systems organization~Redundancy</concept_desc>
  <concept_significance>300</concept_significance>
 </concept>
 <concept>
  <concept_id>10010520.10010553.10010554</concept_id>
  <concept_desc>Computer systems organization~Robotics</concept_desc>
  <concept_significance>100</concept_significance>
 </concept>
 <concept>
  <concept_id>10003033.10003083.10003095</concept_id>
  <concept_desc>Networks~Network reliability</concept_desc>
  <concept_significance>100</concept_significance>
 </concept>
</ccs2012>
\end{CCSXML}

\ccsdesc[500]{Computer systems organization~Embedded systems}
\ccsdesc[300]{Computer systems organization~Redundancy}
\ccsdesc{Computer systems organization~Robotics}
\ccsdesc[100]{Networks~Network reliability}

%%
%% Keywords. The author(s) should pick words that accurately describe
%% the work being presented. Separate the keywords with commas.
\keywords{datasets, neural networks, gaze detection, text tagging}

%% A "teaser" image appears between the author and affiliation
%% information and the body of the document, and typically spans the
%% page.
\begin{teaserfigure}
  \includegraphics[width=\textwidth]{sampleteaser}
  \caption{Seattle Mariners at Spring Training, 2010.}
  \Description{Enjoying the baseball game from the third-base
  seats. Ichiro Suzuki preparing to bat.}
  \label{fig:teaser}
\end{teaserfigure}

%%
%% This command processes the author and affiliation and title
%% information and builds the first part of the formatted document.
\maketitle

\section{Introduction}\label{sec:intro}

\todo


\chapter{Overview}

The prevalent uses of JavaScript in web programming have revealed
security vulnerability issues of JavaScript applications,
which emphasizes the need for JavaScript analyzers to detect such issues.
Recently, researchers have proposed several analyzers of JavaScript programs
and some web service companies have developed various JavaScript engines.
However, unfortunately, most of the tools are not documented well,
thus it is very hard to understand and modify them.
Or, such tools are often not open to the public.


In this specification, we present formal specification and implementation
of \safe, a scalable analysis framework for ECMAScript, developed for the JavaScript research community.
This is the very first attempt to provide both formal specification and
its open-source implementation for JavaScript,
compared to the existing approaches focused on only one of them.
To make it more amenable for other researchers to use our framework,
we formally define three kinds of intermediate representations for JavaScript
used in the framework,
and we provide formal specifications of translations between them.
To be adaptable for adventurous future research including modifications
in the original JavaScript syntax,
we actively use open-source tools to automatically generate parsers and some intermediate
representations.
To support a variety of program analyses in various compilation phases,
we design the framework to be as flexible, scalable, and pluggable as possible.
Finally, our framework is publicly available,
and some collaborative research using the framework are in progress.


\section{Formalization}

In this section, we formally define the dynamic shortcut over the abstract
interpretation.


\subsection{Concrete Semantics}
\[
  \prog ::= (\lab: \inst)^*\\
\]
A program $\prog$ is a sequence of labelled instructions.  We represent the
semantics of a program $\prog$ as a state transition system $(\stset, \trans,
\istset)$.  A state $\st \in \stset = \labset \times \memset$ is a pair of a
label and a memory and represents a status of the program $\prog$.  A memory
$\mem \in \memset = \locset \finmap \valset$ is a finite mapping from locations
to values. A program starts with an initial state in $\istset$.  The transition
relation $\trans \subseteq \stset \times \stset$ describes how states are
transformed to other states.

A \textit{collecting semantics} $\sem{\prog} = \{ \st \in \stset \mid \ist \in
\istset \wedge \ist \trans^* \st \}$ consists of reachable states from initial
states of the program $\prog$.  We could calculate it using the \textit{transfer
function} $\transfer: \dom \rightarrow \dom$ as follows:
\[
  \sem{\prog} = \underset{n \rightarrow \infty}{\lim}{\transfer^n(\ielem)}\\
  \qquad
  \transfer(\elem) = \elem \join \step(\elem)\\
\]
The \textit{concrete domain} $\dom = \powerset{\stset}$ is a complete lattice
with $\cup$, $\cap$, and $\subseteq$ as its join($\join$), meet($\meet$), and
partial order($\order$) operators.  The initial states are $\ielem = \istset$.
The \textit{one-step execution} $\step: \dom \rightarrow \dom$ transforms states
using the transition relation $\trans$: $\step(\elem) = \{ \st' \mid \st \in
\elem \wedge \st \trans \st' \}$.


\subsection{Abstract Interpretation}
The abstract interpretation over-approximate the transfer $\transfer$ to the
\textit{abstract transfer function} $\abstransfer: \absdom \rightarrow \absdom$
to get the \textit{abstract semantics} $\abssem{\prog}$ in finite iterations as
follows:
\[
    \abssem{\prog} = \underset{n \rightarrow
    \infty}{\lim}{(\abstransfer)^n(\iabselem)}\\
\]
We define a \textit{state abstraction} $\dom \galois{\alpha}{\gamma} \absdom$ as
a Galois connection between the concrete domain $\dom$ and an abstract domain
$\absdom$ with a \textit{concretization function} $\gamma$ and a
\textit{abstraction function} $\alpha$.  The initial abstract state $\iabselem
\in \absdom$ represents an abstraction of the initial state set; $\ielem
\subseteq \gamma(\iabselem)$.  The abstract transfer function $\abstransfer:
\absdom \rightarrow \absdom$ is defined as $\abstransfer(\abselem) = \abselem
\join \absstep(\abselem)$ with an \textit{abstract one-step execution}
$\absstep: \absdom \rightarrow \absdom$.  For the sound state abstraction, the
join operator and the abstract one-step execution should satisfy the following
conditions:
\begin{itemize}
  \item $\forall \abselem_0, \abselem_1 \in \absdom. \; \gamma(\abselem_0) \cup
    \gamma(\abselem_1) \subseteq \gamma(\abselem_0 \join \abselem_1)$
  \item $\forall \abselem \in \absdom. \; \absstep \circ \gamma(\abselem) \subseteq
    \gamma \circ \absstep(\abselem)$
\end{itemize}


\subsection{Analysis Sensitivity}

Abstract interpretation is often defined with \textit{analysis sensitivity} to
increase the precision of static analysis.  A sensitive abstract domain
$\sabsdom: \viewset \rightarrow \absdom$ is defined with a \textit{view
abstraction} $\viewmap: \viewset \rightarrow \dom$ that provides multiple points
of views for reachable states during static analysis.  It maps a finite number
of views $\viewset$ to sets of states $\dom$. Each view $\view \in \viewset$
represents a set of states $\viewmap(\view)$.
A \textit{sensitive state abstraction} $\dom
\galois{\alpha_\viewmap}{\gamma_\viewmap} \sabsdom$ is a Galois connection between
the concrete domain $\dom$ and the sensitive abstract domain $\sabsdom$ with the
following concretization function:
\[
  \gamma_\viewmap(\abselem) = \{ \st \in \stset \mid \forall \view \in \viewset.
  \; \st \in \viewmap(\view) \Rightarrow \st \in \gamma \circ \sabselem(\view) \}
\]

With analysis sensitivities, the abstract one-step execution $\sabsstep:
\sabsdom \rightarrow \sabsdom$ is defined as follows:
\[
  \sabsstep(\sabselem) = \lambda \view \in \viewset. \; \underset{\view' \in
  \viewset}{\bigjoin}{\viewtrans{\view'}{\view} \circ \sabselem(\view')}
\]
where $\viewtrans{\view'}{\view}: \absdom \rightarrow \absdom$ is the abstract
semantics of a \textit{view transition} from a view $\view'$ to another view
$\view$.  It should satsify the following condition for the soundness of the
analysis:
\[
  \forall \abselem \in \absdom. \; \step(\gamma(\abselem) \cap \viewmap(\view'))
  \cap \viewmap(\view) \subseteq \gamma \circ
  \viewtrans{\view'}{\view}(\abselem)
\]

One of the most widely-used sensitivity techniques is \textit{flow sensitivity}
using a flow sensitive view abstraction $\fsviewmap: \labset \rightarrow \dom$.
It discriminates states using their labels as follows:
\[
  \forall \lab \in \labset. \; \fsviewmap(\lab) = \{ \st \in \stset \mid \st =
  (\lab, \_) \}
\]


\subsection{Abstract Counting}

We define a \textit{memory abstraction} $\powerset{\memset}
\galois{\alpha_\memset}{\gamma_\memset} \absmemset$ as a Galois connection
between sets of concrete memories and abstract memories.  An abstract memory
$\absmem \in \absmemset: \abslocset \finmap \absvalset$ is a finite mapping from
abstract locations to abstract values.  The \textit{strong update} in memory
abstraction is important to increase the precision of static analysis.  For
sound analysis, a memory update for a specific abstract location should join old
abstract values with new abstract values, called as a \textit{weak update}.
It degrades the analysis precision because of the sprious values.  To overcome
such degradation, researchers proposed the abstract
counting~\cite{abstract-gc-counting, revisit-recency} to track how many times an
abstract location has been allocated and to apply strong updates to singleton
abstract locations.

We extend abstract states with an abstract counting operator $\abscount{-}:
\abslocset \rightarrow \abscountset = \{ \abszero, \absone, \absmany \}$, which
is a function that takes an abstract location and returns its abstract count;
$\abszero$ denotes that it have never been allocated, $\absone$ once, and
$\absmany$ more than or equals to twice.


\subsection{Lazy Concrete Interpretation}

We define a \textit{lazy concrete interpretation} by extending the transition
relation $\trans$ as a symbolic transition relation $\symbtrans$ on abstract
states. First, we extends the concrete states $\stset$ to \textit{symbolic
states} $\symbstset = (\labset \times \symbmemset) \cup \{ \excst \}$ with
symbolic memories $\symbmemset$ consisting of symbolic values $\symb \in
\symbset$ and a special exception state $\excst$. Each symbolic value denotes an
abstract value cannot be concretized to a singleton value.  A symbolic state
\textit{alwyas} has a symbolic transition relation with a single symbolic state.
When a symbolic state has a non-determinism or accesses symbolic values, it has
the relation with the exception state: $\symbst \symbtrans \excst$.  Thus, the lazy
concrete interpretation does not care about the non-deterministic semantics or
symbolic values.

We define a Galois connection $\symbstset
\galois{\alpha_\symbset}{\gamma_\symbset} \absdom$ to freely convert an abstract
state to the corresponding symbolic state and vice versa.  In the concretization
function $\gamma_\symbset$, if an abstract value $\absval$ represents a
singleton value, the function transforms the abstract value to the
corresponding concrete value.  Otherwise, it keeps the abstract value as a
symbolic value in the result symbolic state.


\subsection{Combined Analysis}

We define the \textit{combined analysis} of the abstract interpretation and the
lazy concrete interpretation by extending the view transition as follows:
\[
  \viewtrans{\view'}{\view}_\symbset(\abselem) = \left\{
    \begin{array}{ll}
      \abselem_0 & \text{if} \; \symbst' \symbtrans \excst\\
      \alpha_\symbset(\symbst) & \text{if} \; \symbst' \symbtrans \symbst
      \wedge\\
      & \phantom{\text{if}} \; \gamma \circ \alpha_\symbset(\symbst)
      \cap \viewmap(\view) \neq \varnothing\\ \bot & \text{otherwise}\\
    \end{array}
  \right.
\]
where $\symbst' = \gamma_\symbset \circ \abselem(\view')$ and $\abselem_0 =
\viewtrans{\view'}{\view}(\abselem)$.  The combined analysis first tries to
convert the current abstract state $\abselem(\view')$ to the corresponding
symbolic state $\symbst'$ and performs the lazy concrete evaluation.  When the
evaluation produces the exception state $\excst$, it utilizes the abstract
interpretation instead of the lazy concrete interpretation.  Otherwise, the lazy
concrete interpretation successfully produces a new symbolic state $\symbst$.
If the symbolic state $\symbst$ is related to the target view $\view$, the
result is the corresponding abstract state $\alpha_\symbset(\symbst)$ of the
symbolic state $\symbst$. Otherwise, the result becomes the bottom $\bot$.


\section{Application: JavaScript}

In this section, we introduce the core language of JavaScript that supports
first-class functions, open objects, and first-class property names, and define
combined analysis of the core language.

\subsection{Core Language of JavaScript}

\begin{figure*}[t]
  \centering

  \fbox{$\st \trans \st$}
  \begin{mathpar}
    \inferrule*[width=0.48\textwidth]
    {
      \prog(\lab) = \refer = \expr\\
      \referrule{\st}{\refer}{\loc}\\
      \exprrule{\st}{\expr}{\val}\\
    }
    {
      \st = (\lab, \mem, \ctxtstack, \addr)
      \trans
      (\labnext(\lab), \mem[\loc \mapsto \val], \ctxtstack, \addr)
    }

    \inferrule*[width=0.48\textwidth]
    {
      \prog(\lab) = \refer = \kwobj\\
      \referrule{\st}{\refer}{\loc}\\
      \addr' = \text{(a fresh object address)}
    }
    {
      \st = (\lab, \mem, \ctxtstack, \addr)
      \trans
      (\labnext(\lab), \mem[\loc \mapsto \addr'], \ctxtstack, \addr)
    }

    \inferrule*[width=0.48\textwidth]
    {
      \prog(\lab) = \refer = \expr_f ( \expr_a )\\
      \referrule{\st}{\refer}{\loc}\\
      \exprrule{\st}{\expr_f}{\fval{x}{\lab_b}}\\
      \exprrule{\st}{\expr_a}{\val_a}\\
      \addr' = \text{(a fresh environment address)}
    }
    {
      \st = (\lab, \mem, \ctxtstack, \addr)
      \trans
      (\lab_b, \mem[(\addr', x) \mapsto \val_a], (\addr, \labnext(\lab), \loc)
      :: \ctxtstack, \addr')
    }

    \inferrule*[width=0.48\textwidth]
    {
      \prog(\lab) = \kwret \; \expr\\
      \exprrule{\st}{\expr}{\val}\\
    }
    {
      \st = (\lab, \mem, (\addr', \lab', \loc) :: \ctxtstack, \addr)
      \trans
      (\lab', \mem[\loc \mapsto \val], \ctxtstack, \addr')
    }

    \inferrule*[width=0.48\textwidth]
    {
      \prog(\lab) = \kwif \; \expr \; \lab'\\
      \exprrule{\st}{\expr}{\kwtrue}\\
    }
    {
      \st = (\lab, \mem, \ctxtstack, \addr)
      \trans
      (\lab', \mem, \ctxtstack, \addr)
    }

    \inferrule*[width=0.48\textwidth]
    {
      \prog(\lab) = \kwif \; \expr \; \lab'\\
      \exprrule{\st}{\expr}{\kwfalse}\\
    }
    {
      \st = (\lab, \mem, \ctxtstack, \addr)
      \trans
      (\labnext(\lab), \mem, \ctxtstack, \addr)
    }
  \end{mathpar}

  \fbox{$\referrule{\st}{\refer}{\loc}$}
  \begin{mathpar}
    \inferrule*[width=0.48\textwidth]
    {}
    {
      \referrule{\st = (\lab, \mem, \ctxtstack, \addr)}{x}{(\addr, x)}\\
    }

    \inferrule*[width=0.48\textwidth]
    {
      \exprrule{\st}{\expr_0}{\addr_0}\\
      \exprrule{\st}{\expr_1}{\val_1}\\
      \val_1 \in \strset\\
    }
    {
      \referrule{\st = (\lab, \mem, \ctxtstack, \addr)}{\expr_0 [ \expr_1
      ]}{(\addr_0, \val_1)}
    }
  \end{mathpar}

  \fbox{$\exprrule{\st}{\expr}{\val}$}
  \begin{mathpar}
    \inferrule*[width=0.48\textwidth]
    {
    }
    {
      \exprrule{\st = (\lab, \mem, \ctxtstack, \addr)}{\pval}{\pval}
    }

    \inferrule*[width=0.48\textwidth]
    {
    }
    {
      \exprrule{\st = (\lab, \mem, \ctxtstack,
      \addr)}{\fval{x}{\lab'}}{\fval{x}{\lab'}}
    }

    \inferrule*[width=0.48\textwidth]
    {
      \referrule{\st}{\refer}{\loc}\\
      \loc \in \mem\\
    }
    {
      \exprrule{\st = (\lab, \mem, \ctxtstack, \addr)}{\refer}{\mem(\loc)}
    }

    \inferrule*[width=0.48\textwidth]
    {
      \referrule{\st}{\refer}{\loc}\\
      \loc \not\in \mem\\
    }
    {
      \exprrule{\st = (\lab, \mem, \ctxtstack, \addr)}{\refer}{\kwundef}
    }

    \inferrule*[width=0.48\textwidth]
    {
      \exprrule{\st}{\expr_1}{\val_1}\\
      \cdots\\
      \exprrule{\st}{\expr_n}{\val_n}\\
    }
    {
      \exprrule{\st = (\lab, \mem, \ctxtstack, \addr)}
      {\op(\expr_1, \cdots, \expr_n)}{\op(\val_1, \cdots, \val_n)}
    }
  \end{mathpar}

  \caption{The transition relation for the core language of JavaScript}
  \label{fig:core-trans-rel}
\end{figure*}

\[
  \begin{array}{ll@{~}c@{~}l}
    \text{Programs} & \prog &::=& (\lab: \inst)^*\\

    \text{Labels} & \lab &\in& \labset\\

    \text{Instructions} & \inst &::=&
    \refer = \expr \mid
    \refer = \kwobj \mid
    \refer = \expr ( \expr ) \mid
    \kwret \; \expr \mid
    \kwif \; \expr \; \lab\\

    \text{References} & \refer &::=&
    x \mid
    \expr [ \expr ]\\

    \text{Expressions} & \expr &::=&
    \pval \mid
    \lambda x. \; \lab \mid
    \refer \mid
    \op(\expr^*)\\
  \end{array}
\]

A program $\prog$ is a sequence of labelled instructions. An instruction $\inst$
is an expression assignment, an object creation, a function call, a return
instruction, or a branch.  An expression $\expr$ is a primitive, a lambda
function, a reference, or an operation between other expressions.  A reference
$\refer$ is a variable or a property access of an object.

\[
  \begin{array}{lr@{~}c@{~}l@{~}c@{~}l}
    \text{States} & \st &\in& \stset &=& \labset \times \memset \times
    \ctxtset^* \times \eaddrset\\
    \text{Memories} & \mem &\in& \memset &=& \locset \rightarrow \valset\\
    \text{Contexts} & \ctxt &\in& \ctxtset &=& \eaddrset \times \labset \times
    \locset\\
    \text{Locations} & \loc &\in& \locset &=& (\eaddrset \times \varset) \uplus
    (\oaddrset \times \strset)\\
    \text{Values} & \val &\in& \valset &=& \pvalset \uplus \oaddrset \uplus
    \fvalset\\
    \text{Primitives} & \pval &\in& \pvalset &=& \strset \uplus \cdots\\
    \text{Addresses} & \addr &\in& \addrset &=& \eaddrset \uplus \oaddrset\\
    \text{Functions} & \fval{x}{\lab} &\in& \fvalset &=& \varset \times
    \labset\\
  \end{array}
\]

States $\stset$ consist of labels $\labset$, memories $\memset$, context stacks
$\ctxtset^*$, and environment addresses.  A memory $\mem \in \memset$ is a
mapping from locations to values.  Context stacks is sequences of contexts
$\ctxtset$ and a context $\ctxt \in \ctxtset$ is a tuple of an environment
address, a return label, and a left-hand side location.  A location $\loc \in
\locset$ is a variable or an object property; a variable location consists of an
environment address and its name, and an object property location consists of an
object address and a string value.  A value $\val \in \valset$ is a primitive,
an address, or a function value.  An address $\addr \in \addrset$ is an
environment address or an object address.  A function value
$\fval{x}{\lab}{\addr} \in \fvalset$ consists a parameter and a body label.  In
the core language, the closed scoping is used for functions for brevity thus
only parameters and local variables are accessible in the function body.

We formulate the concrete semantics of the core language as described in
Figure~\ref{fig:core-trans-rel}.  The transition relation between concrete
states is defined with the semantics of references and expressions using two
different forms \fbox{$\referrule{\st}{\refer}{\loc}$} and
\fbox{$\exprrule{\st}{\expr}{\val}$}, respectively.  The special value
$\kwundef$ denotes an undefined value and it is produced when the program access
an unknown location.


\subsection{Abstract Semantics}

\todo


\subsection{Abstract Counting}

We define a \textit{memory abstraction} $\powerset{\memset}
\galois{\alpha_\memset}{\gamma_\memset} \absmemset$ as a Galois connection
between sets of concrete memories and abstract memories.  An abstract memory
$\absmem \in \absmemset: \abslocset \finmap \absvalset$ is a finite mapping from
abstract locations to abstract values.  The \textit{strong update} in memory
abstraction is important to increase the precision of static analysis.  For
sound analysis, a memory update for a specific abstract location should join old
abstract values with new abstract values, called as a \textit{weak update}.
It degrades the analysis precision because of the sprious values.  To overcome
such degradation, researchers proposed the abstract
counting~\cite{abstract-gc-counting, revisit-recency} to track how many times an
abstract location has been allocated and to apply strong updates to singleton
abstract locations.

We extend abstract states with an abstract counting operator $\abscount{-}:
\abslocset \rightarrow \abscountset = \{ \abszero, \absone, \absmany \}$, which
is a function that takes an abstract location and returns its abstract count;
$\abszero$ denotes that it have never been allocated, $\absone$ once, and
$\absmany$ more than or equals to twice.


\section{Evaluation}\label{sec:eval}

We developed $\tool$ to implement dynamic shortcut technique for JavaScript
static analysis (Section~\ref{sec:javascript}).  Our tool is an extension of
a state-of-the-art JavaScript static analyzer SAFE 2.0~\cite{safe2} with the
dynamic analyzer Jalangi 2~\cite{jalangi}.

\todo - Explain the detail of implementation

We evaluated our tool based on the following research questions:
\begin{itemize}
  \item \textbf{RQ1) Analysis Speed-up:} How much analysis time is reduced by
    adding dynamic shortcut to static analysis?
  \item \textbf{RQ2) Precision Improvement:} How much analysis precision is
    improved by replacing the manual modeling with dynamic shortcut?
  \item \textbf{RQ3) Opaque Function Coverage:} How many opaque functions are
    covered by dynamic shortcut without using manual modeling?
\end{itemize}
We performed our experiments on an Ubuntu machine equipped with 4.2GHz Quad-Core
Intel Core i7 and 64GB of RAM.


\subsection{Analysis Speed-up}

\begin{table}
  \caption{Analysis of 306 tests of Lodash 4.}
  \label{table:conc-test}
  \vspace*{-1em}
  \centering
  \[
    \begin{array}{c?r|r|r}
      &
      \multicolumn{1}{c|}{\text{SAFE (ms)}} &
      \multicolumn{1}{c|}{\tool \text{ (ms)}} &
      \multicolumn{1}{c}{\text{Speed Up}}\\\hline\hline
      \text{avg.}       & \inred{168,399} & \inred{1,811} & \inred{116.38}\\\hline
      \text{min.}       & \inred{ 37,095} & \inred{  514} & \inred{ 12.81}\\\hline
      \text{max.}       & \inred{299,375} & \inred{2,986} & \inred{489.67}\\\hline\hline
      \text{\# success} & \inred{    274} & \inred{  306} &
    \end{array}
  \]
  \vspace*{-1em}
\end{table}

We targeted official 306 tests of Lodash 4\cite{lodash} used in motivating
examples (Section~\ref{sec:motivation}).  Recent papers for JavaScript static
analysis techniques~\cite{value-refinement, value-partitioning} also evaluated
their techniques based on them.  Table~\ref{table:conc-test} shows that our tool
successfully analyzes \inred{all 306} tests in \inred{514 ms} on average while
SAFE only analyzes \inred{274} out of 306 tests within 5 minute timeout.  For
\inred{274} tests analyzable by SAFE, our tool is at least \inred{12.81x}, at
most \inred{489.67x}, and \inred{116.38x} on average faster than SAFE.

However, all tests do not contain any non-deterministic values thus they could
be analyzed via only dynamic analysis without static analysis.  In fact, our
tool covers all program parts via only one dynamic shortcut without using
abstract semantics of static analysis.  Thus, we modified 306 official tests to
include abstract values by lifting each literal to the corresponding typed
value.  For example, the numerical literal \jscode{42} will be lifted to the
abstract value $\top_{\code{num}}$, which represents the whole numerical values.

% abstract versions
% of official 
% 
% performs only dynamic shortcut
% 
% by only performing dynamic shortcut for
% whole program points.  Moreover, they analyzed 
% 
% , which is a modern JavaScript library
% delivering modularity, performance, and extras.  It is the first 
% 
% 
% For motivating examples, we excerpt the \jscode{concat} function in
% Figure~\ref{fig:concat} from Lodash library~\cite{lodash} (v4.17.20), which is
% the most popular npm package\footnote{https://www.npmjs.com/browse/depended}
% and \inred{124,562} npm packages have dependency with it.  The \jscode{concat}
% 


\section{Rights Information}

Authors of any work published by ACM will need to complete a rights
form. Depending on the kind of work, and the rights management choice
made by the author, this may be copyright transfer, permission,
license, or an OA (open access) agreement.

Regardless of the rights management choice, the author will receive a
copy of the completed rights form once it has been submitted. This
form contains \LaTeX\ commands that must be copied into the source
document. When the document source is compiled, these commands and
their parameters add formatted text to several areas of the final
document:
\begin{itemize}
\item the ``ACM Reference Format'' text on the first page.
\item the ``rights management'' text on the first page.
\item the conference information in the page header(s).
\end{itemize}

Rights information is unique to the work; if you are preparing several
works for an event, make sure to use the correct set of commands with
each of the works.

The ACM Reference Format text is required for all articles over one
page in length, and is optional for one-page articles (abstracts).




\section{Conclusion}\label{sec:conclusion}
We presented a novel technique for JavaScript static analysis using \textit{dynamic shortcuts}.
It can significantly accelerate static
analysis by freely leveraging high performance of dynamic analysis for
concretely executable program parts.  To maximize such benefits,
we proposed \textit{sealed symbolic execution}, which performs
concrete execution using sealed symbolic values for abstract values.
We formally defined static analysis using dynamic shortcuts in the
abstract interpretation framework and proved its soundness and termination.
We developed $\tool$ as a prototype implementation of the proposed approach
by extending a combination of the state-of-the-art static and dynamic
analyzers SAFE and Jalangi.  Our tool accelerates the speed
of static analysis 19.96$\x$ for original tests and 6.30$\x$ for
abstracted tests of Lodash 4 library.  Moreover, it detects 6 more
dead branches by using sealed symbolic execution instead of
manual modeling for 12 opaque functions on average.



%%
%% The next two lines define the bibliography style to be used, and
%% the bibliography file.
\bibliographystyle{ACM-Reference-Format}
\bibliography{ref}

\end{document}
\endinput
%%
%% End of file `sample-sigplan.tex'.
