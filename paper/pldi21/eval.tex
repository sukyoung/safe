\section{Evaluation}\label{sec:eval}

We evaluate $\tool$ using the following research questions:
\begin{itemize}
\item \textbf{RQ1) Analysis Speed-up:} How much analysis time is reduced by
using dynamic shortcuts?
\item \textbf{RQ2) Precision Improvement:} How much analysis precision is
improved by using dynamic shortcuts?
% instead of manual modeling?
\item \textbf{RQ3) Opaque Function Coverage:} How many opaque functions are
covered only by dynamic shortcuts?
% without using manual modeling?
\end{itemize}
We selected the official 306 tests of Lodash 4
(v.4.17.20)\footnote{https://github.com/lodash/lodash/blob/4.17.20/test/test.js}
used in the examples in Section~\ref{sec:motivation} as our evaluation target.
Recent work~\cite{value-refinement,
value-partitioning} also used the tests to evaluate their techniques.
Among them, we filtered out 37 tests that use JavaScript language
features SAFE does not support such as dynamic code generation using
\njscode{Function}, getters and setters, and browser-specific features like $\jscode{__proto__}$.
Thus, we used 269 out of 306 tests for the evaluation of $\tool$.
We performed our experiments on a Ubuntu machine
equipped with 4.2GHz Quad-Core Intel Core i7 and 64GB of RAM.


\subsection{Analysis Speed-up}

\begin{figure}[t]
  \centering
  \includegraphics[width=\linewidth]{img/conc-analysis-time}
  \vspace*{-1.5em}
  \caption{Analysis time for Lodash 4 \textit{original} tests without (no-DS)
  and with (DS) dynamic shortcuts within 5 minutes}
  \label{fig:conc-analysis-time}
  \vspace*{-1.5em}
\end{figure}

\begin{figure}[t]
  \centering
  \includegraphics[width=\linewidth]{img/abs-analysis-time}
  \vspace*{-1.5em}
  \caption{Analysis time for Lodash 4 \textit{abstracted} tests without (no-DS)
  and with (DS) dynamic shortcuts within 5 minutes}
  \label{fig:abs-analysis-time}
  \vspace*{-1.5em}
\end{figure}

To evaluate the effectiveness of using dynamic shortcuts, we performed static
analysis of 269 Lodash 4 tests with and without dynamic shortcuts.
Figure~\ref{fig:conc-analysis-time} depicts cumulative distribution charts for
their analysis time and a box plot in a logarithmic scale for speed up after
applying dynamic shortcuts.  In the upper chart, the $x$-axis is time and the
$y$-axis shows the number of tests within the time.  While the baseline analysis
(no-DS) finished analysis of 200 out of 269 tests within 5 minutes, our tool
(DS) finished analysis of 263 tests using dynamic shortcuts.  For finished
tests, the average analysis time is 49.57 seconds for no-DS and 2.78 seconds for
DS.  Since only few tests analyzed in 2 to 5 minutes (8 tests for no-DS
and no tests for DS), we only show the results within 2 minutes on the
cumulative distribution charts to show the difference between their results more
clearly.  Among 200 tests analyzed by no-DS, two tests are timeout in DS, thus
198 tests are analyzable by both analyzers. For them, we depict the box plot for
analysis speed up by dynamic shortcuts.  It shows that DS
outperforms no-DS up to 54.98$\x$ and 19.96$\x$ on
average.  Only for one test using $\jscode{_.sample}$, which
randomly samples a value from a given array, DS showed
0.90$\x$ speed of no-DS due to 24 times uses of dynamic shortcuts.

Note that since most tests use concrete values instead of
non-deterministic inputs, they can be analyzed by a few number of dynamic shortcuts.
In fact, among 269 tests, 262 tests are analyzed
by a single dynamic shortcut without using abstract semantics.
However, in real-world JavaScript programs, arguments of library
functions may include non-deterministic inputs.
To evaluate $\tool$ in a real-world setting,
we modified the tests to use abstract values.
We made abstract values by randomly selecting literals and replacing
one of them with its corresponding abstract value.
For example, if we select a numeric literal \jscode{42}, we modified it to the abstract numeric value
$\top_{\code{num}}$, which represents all the numeric values.
In the remaining section, we evaluated $\tool$ using the \textit{original} tests
and the \textit{abstracted} tests.

For abstracted tests as well, DS outperformed no-DS.
Figure~\ref{fig:abs-analysis-time} shows the analysis time of the abstracted tests.
Only four tests for no-DS and nine tests for DS are finished between 2 to 5 minutes.
Among 269 abstracted tests, no-DS finished analysis of 158 tests within 5 minutes,
but DS finished analysis of 167 tests.  For finished tests, the average analysis
time is 44.88 seconds for no-DS and 42.28 seconds for DS. Among 158 tests analyzed by no-DS, DS
timed-out for 15 tests.  For 143 tests analyzable by both analyzers,
DS outperformed no-DS up to 50.60$\x$ and 6.30$\x$ on average.

\begin{figure}[t]
  \centering
  \includegraphics[width=\linewidth]{img/abs-analysis-ratio}
  \vspace*{-1.5em}
  \caption{Analysis time ratio for 143 \textit{abstracted} tests}
  \label{fig:abs-analysis-ratio}
  \vspace*{-1.5em}
\end{figure}

\begin{table*}[t]
  \caption{Number of original (orig.) and abstracted (abs.) tests using dynamic shortcuts
only for each JavaScript built-in library}
  \label{table:func-replace}
  \vspace*{-1em}
  \centering
  \scriptsize
  \[
\quad
    \begin{array}{c|l|c|c?c|l|c|c?c|l|c|c}

      \myhead{Object}       {Function}        {\# Replaced}

      \mysutf{1}{}          {Array    }  {203 / 203}{ 90 / 116} & \mysutf{1}{}             {String     }  { 20 /  20}{  9 /  10} & \mysutf{1}{}       {Object        }  {263 / 263}{151 / 167} \mylinefff
      \mysutf{1}{}          {new Array}  {  0 /   0}{  0 /  10} & \mysutf{1}{}             {toString   }  {  0 /   0}{  0 /  33} & \mysutf{1}{}       {new Object    }  {  0 /   0}{  0 /   3} \mylinefff
      \mysutf{1}{}          {isArray  }  {263 / 263}{144 / 167} & \mysutf{1}{}             {valueOf    }  {  0 /   0}{  0 /  12} & \mysutf{1}{}       {getPrototypeOf}  { 56 /  56}{ 24 /  29} \mylinefff
      \mysutf{1}{}          {concat   }  {263 / 263}{163 / 167} & \mysutt{1}{}             {charAt     }  {  7 /   7}{  5 /   5} & \mysutf{1}{}       {create        }  {263 / 263}{164 / 167} \mylinefff
      \mysutf{1}{}          {join     }  {263 / 263}{166 / 167} & \mysutf{1}{}             {charCodeAt }  { 15 /  15}{  3 /   4} & \mysutf{1}{Object} {defineProperty}  {263 / 263}{160 / 167} \mylinefff
      \mysutf{1}{}          {pop      }  { 25 /  25}{  9 /  13} & \mysutt{1}{}             {indexOf    }  {  2 /   2}{  1 /   1} & \mysbtt{1}{}       {freeze        }  {  1 /   1}{  1 /   1} \mylinefff
      \mysutf{2}{Array}     {push     }  {263 / 263}{150 / 167} & \mysutf{1}{String}       {match      }  { 25 /  25}{ 11 /  13} & \mysutf{1}{}       {keys          }  {263 / 263}{161 / 167} \mylinefff
      \mysutt{1}{}          {reverse  }  { 10 /  10}{  3 /   3} & \mysutf{1}{}             {replace    }  { 55 /  55}{ 19 /  26} & \mysutf{1}{}       {toString      }  {263 / 263}{129 / 167} \mylinefff
      \mysutf{1}{}          {shift    }  {  2 /   2}{  0 /   1} & \mysutf{1}{}             {slice      }  {263 / 263}{165 / 167} & \mysutf{1}{}       {hasOwnProperty}  {263 / 263}{161 / 167} \mylinefft
      \mysutf{1}{}          {slice    }  {263 / 263}{164 / 167} & \mysutf{1}{}             {split      }  {  5 /   5}{  0 /   1} & \mysutf{1}{}       {parseInt}        {  1 /   1}{  0 /   1} \mylinefff
      \mysutf{1}{}          {sort     }  { 69 /  69}{ 26 /  28} & \mysutf{1}{}             {substring  }  {214 / 214}{ 91 / 121} & \mysutf{1}{Global} {isNaN   }        { 15 /  15}{  9 /  25} \mylinefff
      \mysutf{1}{}          {splice   }  { 25 /  25}{  6 /  11} & \mysutf{1}{}             {toLowerCase}  {214 / 214}{ 90 / 121} & \mysbtt{1}{}       {isFinite}        {  3 /   3}{  1 /   1} \mylinefft
      \mysutt{1}{}          {unshift  }  {  2 /   2}{  1 /   1} & \mysutf{1}{}             {toUpperCase}  { 10 /  10}{  4 /   6} & \mysbtt{1}{}       {RegExp    }      {263 / 263}{167 / 167} \mylineftf
      \mysutf{1}{}          {indexOf  }  { 94 /  94}{ 22 /  47} & \mysutt{2}{Boolean}      {Boolean}      {  3 /   3}{  2 /   2} & \mysutf{2}{RegExp} {new RegExp}      {  0 /   0}{  0 /   1} \mylinefff
      \mysutf{1}{}          {every    }  { 92 /  92}{ 23 /  32} & \mysutf{1}{}             {valueOf}      {  0 /   0}{  0 /   6} & \mysbtt{1}{}       {exec      }      {263 / 263}{167 / 167} \mylinettf
      \mysutt{1}{}          {ceil }      { 36 /  36}{  8 /   8} & \mysutf{2}{Number}       {Number }      {  2 /   2}{  1 /   2} & \mysutf{1}{}       {test      }      {263 / 263}{158 / 167} \mylinefft
      \mysuff{1}{}          {floor}      { 16 /  17}{  4 /   5} & \mysutf{1}{}             {valueOf}      {  0 /   0}{  0 /  16} & \mysutf{1}{}       {Error         }  {  1 /   1}{  0 /   1} \mylineftf
      \mysutf{1}{Math}      {max  }      {263 / 263}{134 / 167} & \mysutt{1}{}             {toString}     {263 / 263}{167 / 167} & \mysutf{2}{Error}  {new Error     }  {  0 /   0}{  0 /   8} \mylinefff
      \mysutf{1}{}          {min  }      { 64 /  64}{ 15 /  48} & \mnsutf{1}{Function}     {apply   }     {263 / 263}{124 / 167} & \mysutf{1}{}       {new RangeError}  {  0 /   0}{  0 /   3} \mylinefff
      \mysutt{1}{}          {pow  }      { 11 /  11}{  5 /   5} & \mysuff{1}{}             {call    }     {262 / 263}{ 50 / 167} & \mysutf{1}{}       {new TypeError }  {  0 /   0}{  0 /   9}
    \end{array}
  \]
  \vspace*{-1em}
\end{table*}

Unlike for the original tests, analysis of 143 abstracted tests invoked
16.05 dynamic shortcuts.  Because taking a dynamic shortcut
requires conversion between abstract states and sealed symbolic values
and their exchanges between the static analyzer and the dynamic analyzer,
using dynamic shortcuts multiple times may incur more performance
overhead than performance benefits by using sealed symbolic execution.
One conjecture is that the communication cost between the static
analyzer and the dynamic analyzer may be proportional to the number of
dynamic shortcuts.

To experimentally evaluate the conjecture, we investigated the relationship between
the communication cost between analyzers and the number of dynamic shortcuts.
For 198 original tests, the communication cost was only
3.51\% compared to the analysis time of no-DS.  However, for 143
abstracted tests, the communication cost was 82.49\% compared to the analysis
time of no-DS.  Figure~\ref{fig:abs-analysis-ratio} presents the
analysis time ratio for 143 abstracted tests.
The $x$-axis represents the time ratio normalized by the total analysis time of
no-DS and the $y$-axis denotes the number of dynamic
shortcuts and the number of corresponding tests.
For all 143 tests, the communication cost (Comm. Cost) is larger than
both the static analysis time (Static) and the dynamic analysis
time (Dynamic).  When dynamic shortcuts are performed less than 10 times,
the communication cost is modest compared with the baseline static
analysis time.  However, the more dynamic shortcuts are performed,
the less the performance benefits by using dynamic shortcuts.
Specifically, when dynamic shortcuts are performed more than
20 times, Comm. Cost is even larger than no-DS.
Based on this evaluation result, we believe that we can leverage
dynamic shortcuts by optimizing the communication cost between
the static analyzer and the dynamic analyzer.

\begin{figure}[t]
  \centering
  \begin{subfigure}[t]{0.48\textwidth}
    \includegraphics[width=\linewidth]{img/conc-precision}
    \vspace*{-1.5em}
    \caption{Reachable branches for 198 \textit{original} tests}
    \label{fig:precision-fail}
  \end{subfigure}
  \begin{subfigure}[t]{0.48\textwidth}
    \includegraphics[width=\linewidth]{img/abs-precision}
    \vspace*{-1.5em}
    \caption{Reachable branches for 143 \textit{abstracted} tests}
    \label{fig:precision-branch}
  \end{subfigure}
  \vspace*{-1em}
  \caption{Branch coverage of analysis without (no-DS) and with (DS) dynamic shortcuts}
  \label{fig:precision}
  \vspace*{-1.5em}
\end{figure}



\subsection{Precision Improvement}

To evaluate the analysis precision improvement of dynamic shortcuts,
we measured the number of branches covered by no-DS and DS.
Because both no-DS and DS are sound,
high (low) branch coverage denotes low (high) analysis precision.

Figure~\ref{fig:precision} depicts the comparison of the analysis
precision between no-DS and DS.  The $x$-axis and the $y$-axis denote
the number of branches covered by no-DS and DS, respectively;
each $\times$ mark denotes each test.  The top and bottom dotted lines
denote the worst and the best precision improvement, respectively, and
the middle solid line denotes the average improvement.
For 198 original tests that are analyzable by both analyzers,
Figure~\ref{fig:precision}(a) shows that dynamic shortcuts
reduced the number of covered branches up to 13.13\%. However, it reduced
the number of branches 0.46\% on average because most original tests
do not use non-determinism. Unlike original tests, for 143
abstracted tests that are analyzable by both analyzers,
Figure~\ref{fig:precision}(b) shows that dynamic shortcuts successfully cut down
the number of covered branches up to 13.65\% and 1.89\% on average.
Thus, on average, dynamic shortcuts removed analysis of 1.44 and 5.97 spurious branches
for original and abstracted tests, respectively.
