\section{Definition of Dynamic Shortcut}\label{sec:def}

In this section, we introduce the formal definition of dynamic shortcut based on
sealed symbolic execution in abstract interpretation.



\subsection{Abstract Interpretation}

We first define the traditional abstract interpretation.


\subsubsection{Concrete Semantics}

We define a program $\prog$ as a state transition system $(\stset, \trans,
\istset)$.  A program starts with an initial state in $\istset$ and the
transition relation $\trans \subseteq \stset \times \stset$ describes how states
are transformed to other states.  A \textit{collecting semantics} $\sem{\prog} =
\{ \st \in \stset \mid \ist \in \istset \wedge \ist \trans^* \st \}$ consists of
reachable states from initial states of the program $\prog$.  We could calculate
it using the \textit{transfer function} $\transfer: \dom \rightarrow \dom$ as
follows:
\[
  \sem{\prog} = \underset{n \rightarrow \infty}{\lim}{\transfer^n(\ielem)}\\
  \qquad
  \transfer(\elem) = \elem \join \step(\elem)\\
\]
The \textit{concrete domain} $\dom = \powerset{\stset}$ is a complete lattice
with $\cup$, $\cap$, and $\subseteq$ as its join($\join$), meet($\meet$), and
partial order($\order$) operators.  The element $\ielem$ denotes the initial
states $\istset$.  The \textit{one-step execution} $\step: \dom \rightarrow
\dom$ transforms states using the transition relation $\trans$: $\step(\elem) =
\{ \st' \mid \st \in \elem \wedge \st \trans \st' \}$.


\subsubsection{Abstract Interpretation}

The abstract interpretation over-approximates the transfer $\transfer$ to the
\textit{abstract transfer function} $\abstransfer: \absdom \rightarrow \absdom$
to get the \textit{abstract semantics} $\abssem{\prog}$ in finite iterations as
follows:
\[
    \abssem{\prog} = \underset{n \rightarrow
    \infty}{\lim}{(\abstransfer)^n(\iabselem)}\\
\]
We define a \textit{state abstraction} $\dom \galois{\alpha}{\gamma} \absdom$ as
a Galois connection between the concrete domain $\dom$ and an abstract domain
$\absdom$ with a \textit{concretization function} $\gamma$ and an
\textit{abstraction function} $\alpha$.  The initial abstract state $\iabselem
\in \absdom$ represents an abstraction of the initial state set; $\ielem
\subseteq \gamma(\iabselem)$.  The abstract transfer function $\abstransfer:
\absdom \rightarrow \absdom$ is defined as $\abstransfer(\abselem) = \abselem
\join \absstep(\abselem)$ with an \textit{abstract one-step execution}
$\absstep: \absdom \rightarrow \absdom$.  For the sound state abstraction, the
join operator and the abstract one-step execution should satisfy the following
conditions:
\begin{itemize}
  \item $\forall \abselem_0, \abselem_1 \in \absdom. \; \gamma(\abselem_0) \cup
    \gamma(\abselem_1) \subseteq \gamma(\abselem_0 \join \abselem_1)$
  \item $\forall \abselem \in \absdom. \; \absstep \circ \gamma(\abselem) \subseteq
    \gamma \circ \absstep(\abselem)$
\end{itemize}


\subsubsection{Analysis Sensitivity}

Abstract interpretation is often defined with \textit{analysis sensitivity} to
increase the precision of static analysis.  A sensitive abstract domain
$\sabsdom: \viewset \rightarrow \absdom$ is defined with a \textit{view
abstraction} $\viewmap: \viewset \rightarrow \dom$ that provides multiple points
of views for reachable states during static analysis.  It maps a finite number
of views $\viewset$ to sets of states $\dom$. Each view $\view \in \viewset$
represents a set of states $\viewmap(\view)$.
A \textit{sensitive state abstraction} $\dom
\galois{\alpha_\viewmap}{\gamma_\viewmap} \sabsdom$ is a Galois connection between
the concrete domain $\dom$ and the sensitive abstract domain $\sabsdom$ with the
following concretization function:
\[
  \gamma_\viewmap(\sabselem) = \{ \st \in \stset \mid \forall \view \in \viewset.
  \; \st \in \viewmap(\view) \Rightarrow \st \in \gamma \circ \sabselem(\view) \}
\]

With analysis sensitivities, the abstract one-step execution $\sabsstep:
\sabsdom \rightarrow \sabsdom$ is defined as follows:
\[
  \sabsstep(\sabselem) = \lambda \view \in \viewset. \; \underset{\view' \in
  \viewset}{\bigjoin}{\viewtrans{\view'}{\view} \circ \sabselem(\view')}
\]
where $\viewtrans{\view'}{\view}: \absdom \rightarrow \absdom$ is the abstract
semantics of a \textit{view transition} from a view $\view'$ to another view
$\view$.  It should satisfy the following condition for the soundness of the
analysis:
\[
  \forall \abselem \in \absdom. \; \step(\gamma(\abselem) \cap \viewmap(\view'))
  \cap \viewmap(\view) \subseteq \gamma \circ
  \viewtrans{\view'}{\view}(\abselem)
\]



\subsection{Dynamic Shortcut}

We define dynamic shortcut by combining the sensitive abstract interpretation
with sealed symbolic execution.


\subsubsection{Sealed Symbolic Execution}

We define \textit{sealed symbolic execution} by extending the transition
relation $\trans$ as a symbolic transition relation $\symbtrans$ on symbolic
states.  First, we augment concrete states $\stset$ to symbolic states
$\symbstset$ by extending values $\valset$ with \textit{sealed symbolic values}
$\symbset$.  We also define the symbolic transition relation $\symbtrans
\subseteq \symbstset \times \symbstset$. We use the notation $\symbtrans^k$
for $k$ repetition of $\symbtrans$, and  $\symbst \symbtrans \excst$ when
$\symbst$ does not have symbolic transitions with any other symbolic states.  We
also define the soundness of sealed symbolic execution as follows:
\begin{definition}{(Soundness)}
  The symbolic transition relation is \textit{sound} when the following
  condition is satisfied:
  \[
    \begin{array}{l}
      \forall \imap \in \imapset. \; \forall \symbst, \symbst' \in \symbstset.\\
      \symbst \symbtrans \symbst' \Rightarrow \instant{\symbst}{\imap} \trans
      \instant{\symbst'}{\imap}
    \end{array}
  \]
  where $\imapset: \symbset \rightarrow \valset$ is instantiation maps from
  symblic values to real values, and $\instant{\symbst}{\imap}$ denotes a state
  produced by replacing each symbolic value $\symb$ to the corresponding value
  $\imap(\symb)$ using the instantiation map $\imap \in \imapset$.
\end{definition}

Note that the main difference of sealed symbolic execution with the traditional
symbolic execution is that it only supports sealed symbolic values instead of
symbolic expressions and path constraints.


\subsubsection{Combined Domain}

We define a \textit{combined domain} $\combdom = \sabsdom \times
\powerset{\symbstset}$ as a pair of a sensitive abstract domain and a set of
sealed symbolic states.   The join operator of combined states is defined in an
element-wise manner:
\[
  \forall (\sabselem, S), ({\sabselem}', S') \in \combdom. \;
  (\sabselem, S) \join ({\sabselem}', S') = (\sabselem \join {\sabselem}', S \cup S')
\]

To freely convert a pair of view and an abstract state to its
corresponding symbolic state and vice versa, we define two domain converters
$\symbstset \galois{\saconverter}{\asconverter} (\viewset \times \absdom)$.  In
the converter $\asconverter$ from abstract states to symbolic states, if an
abstract value $\absval$ represents a singleton value, the function transforms
the abstract value to the corresponding concrete value.  Otherwise, it converts
the abstract value as a symbolic value in a result symbolic state.  Two
converters convert given elements without loss of information:
\[
  \saconverter \circ \asconverter = \asconverter \circ \saconverter = \identity
\]
where $\identity$ denotes the identity function.

An analysis element $\aelem \in \aelemset = (\viewset \times \absdom) \uplus
\symbstset$ is either a pair of a view and an abstract state or a sealed
symbolic state.  We define a $\atriage$ function using two converters between
analysis elements as follows:
\begin{definition}[$\atriage$]
  The triage function $\atriage: \aelemset \rightarrow \aelemset$ for analysis
  elements is defined as
  \[
    \atriage(\aelem) = \left\{
      \begin{array}{ll}
        \asconverter(\aelem)
        & \text{if} \; \aelem = (\view, \abselem) \wedge \checker(\aelem)\\
        \saconverter(\aelem)
        & \text{if} \; \aelem = \symbst \wedge \symbst \symbtrans \symbst'\\
        \aelem
        & \text{Otherwise}
      \end{array}
    \right.
  \]
  where the function $\checker$ decides whether to perform sealed symbolic
  execution instead of abstract interpretation.
\end{definition}
Using the $\atriage$ function for analysis elements, we defined the $\triage$
function for combined states:
\begin{definition}[$\triage$]
  The triage function $\triage: \combdom \rightarrow \combdom$ for combined
  states is defined as
  \[
    \triage((\sabselem, S)) = \left(
      \lambda \view. \;  \bigjoin \{ \abselem \mid (\view, \abselem) \in E \},
      E \cap \symbstset
    \right)
  \]
  where
  \[
    E = \dot{\atriage}(\{ (\view, \abselem) \mid \view \in \viewset \wedge
    \sabselem(\view) = \abselem \} \cup S)
  \]
\end{definition}
The dot notation $\dot{f}$ denotes the element-wise extended function of a
function $f$.

We could configure when the sealed symbolic execution or the abstract
interpretation is performed based on the definition of the $\checker$ function.
For example, we could define the $\checker$ that passes only function entry,
call, and exit points to perform conversions in a function level.  For the
soundness and the termination of the the abstract interpretation with dynamic
shortcuts, it should satisfy the following condition:
\begin{theorem}\label{theorem:shortcut}
  The abstract interpretation with dynamic shortcut is sound and terminates in a
  finite time if its abstract semantics is sound and $\checker$ satisfies the
  following condition:
  \[
    \checker((\view, \abselem)) \Rightarrow \asconverter((\view, \abselem))
    \symbtrans^k \excst \wedge 1 < k \leq N
  \]
  where $N$ is a pre-defined maximum length of the sealed symbolic execution.
\end{theorem}

In the remaining section, we will prove this theorem using the Kleene's
fixed-point theorem with the finite height of abstract domains.
