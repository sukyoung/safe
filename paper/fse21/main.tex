%%
%% This is file `sample-sigconf.tex',
%% generated with the docstrip utility.
%%
%% The original source files were:
%%
%% samples.dtx  (with options: `sigconf')
%% 
%% IMPORTANT NOTICE:
%% 
%% For the copyright see the source file.
%% 
%% Any modified versions of this file must be renamed
%% with new filenames distinct from sample-sigconf.tex.
%% 
%% For distribution of the original source see the terms
%% for copying and modification in the file samples.dtx.
%% 
%% This generated file may be distributed as long as the
%% original source files, as listed above, are part of the
%% same distribution. (The sources need not necessarily be
%% in the same archive or directory.)
%%
%% The first command in your LaTeX source must be the \documentclass command.
\documentclass[sigconf,review,anonymous]{acmart}
\acmConference[ESEC/FSE 2021]{The 29th ACM Joint European Software Engineering Conference and Symposium on the Foundations of Software Engineering}{23 - 27 August, 2021}{Athens, Greece}

\usepackage{float}
%\usepackage{amssymb}
\usepackage{amsmath,amsfonts}
\usepackage[ruled, vlined]{algorithm2e}
\usepackage{graphicx}
\usepackage{textcomp}
\usepackage{xcolor}
\usepackage{listings}
\usepackage{caption}
\usepackage{subcaption}
\usepackage{multirow}
\usepackage{booktabs}
\usepackage{makecell}
\usepackage{galois}
\usepackage{mathpartir}
\usepackage{bussproofs}
\usepackage{mathtools}
\usepackage{colortbl}
\usepackage{hhline}
\usepackage{stmaryrd}

%% Rights management information.  This information is sent to you
%% when you complete the rights form.  These commands have SAMPLE
%% values in them; it is your responsibility as an author to replace
%% the commands and values with those provided to you when you
%% complete the rights form.
\setcopyright{acmcopyright}
\copyrightyear{2018}
\acmYear{2018}
\acmDOI{10.1145/1122445.1122456}

%% These commands are for a PROCEEDINGS abstract or paper.
\acmConference[Woodstock '18]{Woodstock '18: ACM Symposium on Neural
  Gaze Detection}{June 03--05, 2018}{Woodstock, NY}
\acmBooktitle{Woodstock '18: ACM Symposium on Neural Gaze Detection,
  June 03--05, 2018, Woodstock, NY}
\acmPrice{15.00}
\acmISBN{978-1-4503-XXXX-X/18/06}


%%
%% Submission ID.
%% Use this when submitting an article to a sponsored event. You'll
%% receive a unique submission ID from the organizers
%% of the event, and this ID should be used as the parameter to this command.
%%\acmSubmissionID{123-A56-BU3}

%%
%% The majority of ACM publications use numbered citations and
%% references.  The command \citestyle{authoryear} switches to the
%% "author year" style.
%%
%% If you are preparing content for an event
%% sponsored by ACM SIGGRAPH, you must use the "author year" style of
%% citations and references.
%% Uncommenting
%% the next command will enable that style.
%%\citestyle{acmauthoryear}

% Basic
\DeclareMathAlphabet{\mathpzc}{T1}{pzc}{m}{it}
\newcommand{\powerset}[1]{\mathcal{P}(#1)}
\newcommand{\imbox}[1]{\mbox{\small \textit{#1}}}
\newcommand{\name}[1]{\textsf{#1}}
\newcommand{\inred}[1]{{\color{red}{#1}}}
\newcommand{\todo}{\inred{TODO}}
\newcommand{\abs}[1]{{#1}^{\#}}
\newcommand{\finmap}{{\xrightarrow[]{\text{fin}}}}

% Tool
\newcommand{\tool}{\inred{TOOL}}

% Keywords
\newcommand{\code}[1]{\texttt{#1}}
\newcommand{\kwif}{\code{if}}
\newcommand{\kwret}{\code{ret}}
\newcommand{\kwobj}{\code{\{\}}}
\newcommand{\kwtrue}{\code{true}}
\newcommand{\kwfalse}{\code{false}}
\newcommand{\kwundef}{\code{undef}}

% Notations
\newcommand{\prog}{P}
\newcommand{\labset}{\mathcal{L}}
\newcommand{\lab}{\mathpzc{l}}
\newcommand{\labnext}{\name{next}}
\newcommand{\op}{\name{op}}
\newcommand{\inst}{i}
\newcommand{\expr}{e}
\newcommand{\refer}{r}

% States
\newcommand{\stset}{\mathbb{S}}
\newcommand{\st}{\sigma}
\newcommand{\istset}{\stset_\iota}
\newcommand{\ist}{\st_\iota}

% Memories
\newcommand{\memset}{\mathbb{M}}
\newcommand{\absmemset}{\abs{\memset}}
\newcommand{\mem}{m}
\newcommand{\absmem}{\abs{\mem}}

% Contexts
\newcommand{\ctxtset}{\mathbb{C}}
\newcommand{\absctxtset}{\abs{\ctxtset}}
\newcommand{\ctxt}{c}
\newcommand{\ctxtstack}{\overline{c}}

% Transition Relations
\newcommand{\trans}{\leadsto}

% Complete Lattice
\newcommand{\join}{\sqcup}
\newcommand{\meet}{\sqcap}
\newcommand{\bigjoin}{\bigsqcup}
\newcommand{\bigmeet}{\bigsqcap}
\newcommand{\order}{\sqsubseteq}

% Concrete Domains
\newcommand{\dom}{\mathbb{D}}
\newcommand{\elem}{d}
\newcommand{\ielem}{\elem_\iota}

% Abstract Domains
\newcommand{\absdom}{\abs{\dom}}
\newcommand{\abselem}{\abs{\elem}}
\newcommand{\iabselem}{\abs{\elem}_\iota}

% Concrete Semantics
\newcommand{\sem}[1]{[\![{#1}]\!]}
\newcommand{\transfer}{F}
\newcommand{\step}{\name{step}}

% Abstract Semantics
\newcommand{\abssem}[1]{\abs{\sem{#1}}}
\newcommand{\abstransfer}{\abs{\transfer}}
\newcommand{\absstep}{\abs{\step}}

% Abstract Sensitivity
\newcommand{\viewmap}{\delta}
\newcommand{\sabsdom}{\absdom_\viewmap}
\newcommand{\sabselem}{\abselem_\viewmap}
\newcommand{\viewset}{\Pi}
\newcommand{\view}{\pi}
\newcommand{\sabsstep}{\absstep_\viewmap}
\newcommand{\viewtrans}[2]{\abssem{#1 \rightarrow #2}}

% Flow Sensitivity
\newcommand{\fs}{\name{FS}}
\newcommand{\fsviewmap}{\viewmap^\fs}

% Locations
\newcommand{\locset}{\mathbb{L}}
\newcommand{\loc}{l}
\newcommand{\abslocset}{\abs\locset}
\newcommand{\absloc}{\abs\loc}

% Variables
\newcommand{\varset}{\mathbb{X}}

% Values
\newcommand{\valset}{\mathbb{V}}
\newcommand{\val}{v}
\newcommand{\pvalset}{\valset_\name{p}}
\newcommand{\pval}{\val_\name{p}}
\newcommand{\fvalset}{\mathbb{F}}
\newcommand{\fval}[2]{\lambda #1. #2}
\newcommand{\strset}{\valset_\name{str}}
\newcommand{\str}{\val_\name{str}}
\newcommand{\absvalset}{\abs{\valset}}
\newcommand{\absval}{\abs{\val}}

% Addresses
\newcommand{\addrset}{\mathbb{A}}
\newcommand{\eaddrset}{\addrset_\name{env}}
\newcommand{\oaddrset}{\addrset_\name{obj}}
\newcommand{\addr}{a}

% Abstract Counting
\newcommand{\abscount}[1]{\abs{\lVert#1\rVert}}
\newcommand{\abscountset}{\abs{\mathbb{N}}}
\newcommand{\abszero}{\abs{0}}
\newcommand{\absone}{\abs{1}}
\newcommand{\absmany}{\abs{\geq\!\!2}}

% Lazy Concrete Interpretation
\newcommand{\symb}{\omega}
\newcommand{\symbset}{\Omega}
\newcommand{\symbstset}{\stset_\symbset}
\newcommand{\symbst}{\st_\symbset}
\newcommand{\symbmemset}{\memset_\symbset}
\newcommand{\symbtrans}{\trans_\symbset}
\newcommand{\excst}{\bot_\st}
\newcommand{\converter}{T}

% Rules
\newcommand{\referrule}[3]{#1 \vdash #2 \rightarrow #3}
\newcommand{\exprrule}[3]{#1 \vDash #2 \Rightarrow #3}

%%
%% end of the preamble, start of the body of the document source.
\begin{document}

%%
%% The "title" command has an optional parameter,
%% allowing the author to define a "short title" to be used in page headers.
\title
[Accelerating JavaScript Static Analysis via Dynamic Shortcuts]
{Accelerating JavaScript Static Analysis\\ via Dynamic Shortcuts}

%%
%% The "author" command and its associated commands are used to define
%% the authors and their affiliations.
%% Of note is the shared affiliation of the first two authors, and the
%% "authornote" and "authornotemark" commands
%% used to denote shared contribution to the research.
\author{Joonyoung Park}
\affiliation{%
  \institution{Korea Advanced Institute of Science and Technology}
  \state{Daejeon}
  \country{South Korea}
}
\email{gmb55@kaist.ac.kr}

\author{Jihyeok Park}
\affiliation{%
  \institution{Korea Advanced Institute of Science and Technology}
  \state{Daejeon}
  \country{South Korea}
}
\email{jhpark0223@kaist.ac.kr}

\author{Dongjun Youn}
\affiliation{%
  \institution{Korea Advanced Institute of Science and Technology}
  \state{Daejeon}
  \country{South Korea}
}
\email{f52985@kaist.ac.kr}

\author{Sukyoung Ryu}
\affiliation{%
  \institution{Korea Advanced Institute of Science and Technology}
  \state{Daejeon}
  \country{South Korea}
}
\email{sryu.cs@kaist.ac.kr}

%%
%% By default, the full list of authors will be used in the page
%% headers. Often, this list is too long, and will overlap
%% other information printed in the page headers. This command allows
%% the author to define a more concise list
%% of authors' names for this purpose.
\renewcommand{\shortauthors}{Park and Park, et al.}

%%
%% The abstract is a short summary of the work to be presented in the
%% article.
\begin{abstract}
Static analyses based on abstract interpretation can be integrated into developer workflow and help developers to find bugs.
However, the dynamic features of JavaScript degrade the scalability and precision of the pure static analyses.
Combining static and dynamic analyses is one of the breakthroughs in JavaScript bug detection.
Existing combined analyses have static boundaries that distinguish which analysis used for which part of a program.
Such boundaries are intuitive but limit chances to optimize transitions between two analyses.
We formally present the dynamic shortcut, a more flexible compound of static and dynamic analyses.
We implement a proof of concept bug detector by extending the existing static analyzer supporting a subset of ECMAScript 5.1.
Our experimental results show that the combined analysis indeed speeds up and produces more precise results.
\end{abstract}


%%
%% The code below is generated by the tool at http://dl.acm.org/ccs.cfm.
%% Please copy and paste the code instead of the example below.
%%
\begin{CCSXML}
<ccs2012>
 <concept>
  <concept_id>10010520.10010553.10010562</concept_id>
  <concept_desc>Computer systems organization~Embedded systems</concept_desc>
  <concept_significance>500</concept_significance>
 </concept>
 <concept>
  <concept_id>10010520.10010575.10010755</concept_id>
  <concept_desc>Computer systems organization~Redundancy</concept_desc>
  <concept_significance>300</concept_significance>
 </concept>
 <concept>
  <concept_id>10010520.10010553.10010554</concept_id>
  <concept_desc>Computer systems organization~Robotics</concept_desc>
  <concept_significance>100</concept_significance>
 </concept>
 <concept>
  <concept_id>10003033.10003083.10003095</concept_id>
  <concept_desc>Networks~Network reliability</concept_desc>
  <concept_significance>100</concept_significance>
 </concept>
</ccs2012>
\end{CCSXML}

\ccsdesc[500]{Computer systems organization~Embedded systems}
\ccsdesc[300]{Computer systems organization~Redundancy}
\ccsdesc{Computer systems organization~Robotics}
\ccsdesc[100]{Networks~Network reliability}

%%
%% Keywords. The author(s) should pick words that accurately describe
%% the work being presented. Separate the keywords with commas.
\keywords{JavaScript, static analysis, dynamic analysis, dynamic shortcut, sealed symbolic execution}

%%
%% This command processes the author and affiliation and title
%% information and builds the first part of the formatted document.
\maketitle

\section{Introduction}\label{sec:intro}

\todo

\section{Motivation}\label{sec:motivation}

\begin{figure}[t]
  \centering
  \begin{subfigure}[t]{0.5\textwidth}
    \begin{lstlisting}[style=myJSstyle]
function concat() {
  var length = arguments.length;
  if (!length) return [];
  var array = arguments[0],
      args  = Array(length - 1),
      index = length;
  while (index--)
    args[index-1] = arguments[index];
  return arrayPush(isArray(array) ?
    copyArray(array) : [array],
    baseFlatten(args, 1));
}
    \end{lstlisting}
    \vspace*{-1em}
    \caption{Lodash's \jscode{concat} function.}
    \label{fig:concat}
  \end{subfigure}
  \begin{subfigure}[t]{0.5\textwidth}
    \begin{lstlisting}[style=myJSstyle,firstnumber=13]
function changeCountry(G) {
  ...
  if (G.selectedVal === "US" && state) {
    // deterministic arguments of `concat`
    state.items = _.concat([["Other", "Other"]],
      WebinarBase.questions.state.items);
    state.selectedVal = _.head(_.head(C.items));
  }
}
    \end{lstlisting}
    \vspace*{-1em}
    \caption{Load the list of states of the United States.}
    \label{fig:changeCountry}
  \end{subfigure}
  \begin{subfigure}[t]{0.5\textwidth}
    \begin{lstlisting}[style=myJSstyle,firstnumber=22]
function getData(e) {
  var option = ... // option of server connection
  post(option).then(function(e) {
    if (e.total_records && e.total_records > 0) {
      // non-deterministic arguments of `concat`
      this.pastEvents =
        _.concat(this.pastEvents, e.events);
      this.total = e.total_records;
    } else this.noPastData = !0
  })
}
    \end{lstlisting}
    \vspace*{-1em}
    \caption{Load more Zoom events from the server.}
    \label{fig:getData}
  \end{subfigure}
  \vspace*{-1em}
  \caption{Motivating Example: Excerpts from Lodash library and JavaScript codes
  in \code{zoom.us} site.}
  \label{fig:example}
  \vspace*{-1em}
\end{figure}

In this section, we will explain the motivation of the dynamic shortcut for
JavaScript static analysis with real-world examples described in
Figure~\ref{fig:example}.  We first describe their program behaviors and then
explain how to utilize dynamic analysis during static analysis of them.

For motivating examples, we excerpt the \jscode{concat} function in
Figure~\ref{fig:concat} from Lodash library~\cite{lodash} (v4.17.20), which is
the most popular npm package\footnote{https://www.npmjs.com/browse/depended}
and \inred{124,562} npm packages have dependency with it.  The \jscode{concat}
function creates a new array concatenating given arrays or values.  It first
checks the length of arguments in line 1-3. Then, it stores the first argument
to \jscode{array} in line 4 and copies the remaining arguments to \jscode{args} in
line 5-8.  In line 9, it checks whether \jscode{array} is an array object with
the built-in function \jscode{isArray}.  If so, it creates a new array by
copying the given array via or initializing with a single value in line 10.
Finally, it flattens \jscode{args} via \jscode{baseFlatten} and pushes the
result to the new array in line 11.

For applications of Lodash, we excerpt two functions \jscode{changeCountry} in
Figure~\ref{fig:changeCountry} and \jscode{getData} in Figure~\ref{fig:getData}
from the \code{zoom.us}~\cite{zoom} site.  The website \code{zoom.us} is
homepage of Zoom, which is a videotelephony software program developed by Zooom
Video Communications and it is ranked as \inred{16th} popular web site according
to Alexa\footnote{https://www.alexa.com/siteinfo/zoom.us} in November 2020.


\paragraph{Dynamic shortcut with concrete values.}
When the given arguments of a function are concrete values, we can perform
dynamic analysis instead of static analysis. For example, the
\jscode{changeCountry} function is invoked when a user selects another country
on the drop-down list in the registration page.  It calls the \jscode{concat}
function to update the drop-down list of states or provinces in 17-18.  However,
when the user selects ``United States of America'' (USA), two arguments are
pre-defined with deterministic values; the first one is an array literal
\jscode{[["Other", "Other"]]} and the second one is an array of pairs of
abbreviations and names of the states defined as follows:
\begin{lstlisting}[style=myJSstyle,numbers=none]
WebinarBase.questions.state.items =
  [["AL","Alabama"], ..., ["WY", "Wyoming"]]
\end{lstlisting}
Moreover, \jscode{this} value is also a concrete value, the Lodash top-level
object \jscode{\_}.  Thus, we could perform dynamic analysis by invoking the
\jscode{concat} function with \jscode{\_} as \jscode{this} value and above two
concrete values as arguments.  It increases performance of static analysis by
skipping the analysis of function call in line 17-18 and utilizing the result of
dynamic analysis.

\begin{figure}[t]
  \begin{subfigure}{0.23\textwidth}
    \[
      \begin{array}{|c|c|}\hline
        \text{Property} & \text{Value}\\\hline
        \top & \symb_\jscode{evt}\\\hline
        \jscode{"length"} & \symb_\jscode{int}\\\hline
      \end{array}
    \]
    \vspace*{-1em}
    \caption{\jscode{this.pastEvents}}
    \label{fig:pastEvents}
  \end{subfigure}
  \begin{subfigure}{0.23\textwidth}
    \[
      \begin{array}{|c|c|}\hline
        \text{Property} & \text{Value}\\\hline
        \jscode{0} & \symb_\jscode{evt}\\\hline
        \cdots & \cdots\\\hline
        \jscode{7} & \symb_\jscode{evt}\\\hline
        \jscode{"length"} & \jscode{8}\\\hline
      \end{array}
    \]
    \vspace*{-1em}
    \caption{\jscode{e.events}}
    \label{fig:events}
  \end{subfigure}
  \vspace*{-1em}
  \caption{Concrete objects for arguments with sealed symbolic values.}
  \label{fig:sealed}
  \vspace*{-1em}
\end{figure}

\paragraph{Dynamic shortcut with abstract values.}
Dynamic analysis is still applicable using \textit{sealed symbolic execution}
even if the arguments are not concrete values.  The \jscode{getData} function is
invoked when clicking the ``Load More'' button to load more Zoom events in
``Webinars \& Events'' page.  For each click, the \jscode{getData} sends a POST
request to the server and receives additional event information \jscode{e} in
line 24.  Then, eight events in \jscode{e.events} are appended to
\jscode{this.pastEvents} using the \jscode{concat} function in line 27-28.
However, the arguments of \jscode{concat} are not deterministic because 1) the
event list stored in \jscode{this.pastEvents} is continuously grown for each
load and 2) also each event stored in \jscode{e.events} are dependent on the
data given from the server.

To perform dynamic analysis with abstract values, we sealed the abstract values
of arguments with symbolic values as described in Figure~\ref{fig:sealed}.  It
contains two symbolic values $\symbevt$ and $\symbint$ that represent any event
objects and integer values, respectively.  Then, dynamic analysis is
successfully performed before copying \jscode{array} via \jscode{copyArray} in
line 10.  First, \jscode{length} stores \jscode{2} and passes the length check
in line 1-2. Then, \jscode{array} points to the same object of
\jscode{this.pastEvents} in line 4, \jscode{args} stores an array with a
single object stored in \jscode{e.events} in 5-8, and the \jscode{isArray}
function returns \jscode{true} for \jscode{array} in line 9.  However, it fails
to perform dynamic analysis for \jscode{copyArray} because the \jscode{length}
property of \jscode{array} is the symbolic value $\symbint$.  Thus, the dynamic
shortcut returns the analysis result and the static analyzer continues to
analyze the program from line 10.  Then, only copying via \jscode{copyArray},
flattening via \jscode{baseFlatten}, and pushing via \jscode{arrayPush} utilize
the abstract semantics.  This is how to utilize sealed symbolic execution to
maximize the part of dynamic analysis during static analysis.


\paragraph{Dynamic shortcut for opaque functions.}

The previous two examples also show that dynamic shortcut can improve the
analysis precision and lessen the effort of modeling the opaque functions.
In line 9, the \jscode{isArray} function is a JavaScript built-in library thus
it is written in a native language of the host environment.  Thus, we need to
manually model its behavior to statically analyze it.  Assume that we model the
\jscode{isArray} function to return the top boolean value that denotes both of
\jscode{true} and \jscode{false}.  If we perform static analysis with this
modeling, both of true branch \jscode{copyArray(array)} and false branch
\jscode{[array]} in line 10 are always analyzed while \jscode{[array]} is never
reachable in the motivating examples.  However, with the dynamic shortcut, the
analyzer utilizes the concrete semantics of \jscode{isArray}.  It returns more
precise result \jscode{true} instead of the top boolean value and it is not
necessary to model the \jscode{isArray} function for static analysis of
motivating examples.

\section{Dynamic Shortcuts}\label{sec:formal}
In this section, we formally define static analysis using dynamic shortcuts by
introducing sealed symbolic execution in the abstract interpretation framework.
We extend the formalization of abstract interpretation of \citet{abs-interp-1977,
abs-interp-1992} and views-based analysis sensitivity of \citet{sens-toplas}.
For dynamic shortcuts, we define sealed symbolic execution with a
sealed symbolic domain and abstract instantiation maps.  To combine
sensitive abstract interpretation and sealed symbolic execution, we define
a combined domain of sensitive abstract domain and sealed symbolic domain and
explain it with a simple example. Finally, we prove the soundness and
termination property of abstract interpretation using the combined domain.


\subsection{Concrete Semantics}

We define a program $\prog$ as a state transition system $(\stset, \trans,
\istset)$.  A program starts with an initial state in $\istset$ and the
transition relation $\trans \subseteq \stset \times \stset$ describes how states
are transformed to other states.  A \textit{collecting semantics} $\sem{\prog} =
\{ \st \in \stset \mid \ist~\in~\istset \wedge \ist \trans^* \st \}$ consists of
reachable states from initial states of the program $\prog$.  We can compute
it using a \textit{transfer function} $\transfer: \dom \rightarrow \dom$ as
follows:
\[
  \sem{\prog} = \underset{n \rightarrow \infty}{\lim}{\transfer^n(\ielem)}\\
  \qquad
  \transfer(\elem) = \elem \join \step(\elem)\\
\]
where the \textit{concrete domain} $\dom = \powerset{\stset}$ is a complete lattice
with $\cup$, $\cap$, and $\subseteq$ as its join($\join$), meet($\meet$), and
partial order($\order$) operators.  The set of states $\ielem$ denotes the
initial states $\istset$.  The \textit{one-step execution} $\step: \dom
\rightarrow \dom$ transforms states using the transition relation $\trans$:
$\step(\elem) = \{ \st' \mid \st \in \elem \wedge \st \trans \st' \}$.

\begin{figure}[t]
  \[
    \begin{array}{r@{~}l@{~}c@{~}l}
      \labdot{0} & \kwif \; (\; \varx \geq 0 \;) & \labdot{1} & \varx = \varx ;\\
                 & \kwelse & \labdot{2} & \varx = -\varx ;\\
      \labdot{3} & \varx = -\varx ; & \labdot{4} \\
    \end{array}
  \]
  \vspace*{-1em}
  \caption{Negation of the absolute value of $\varx$}
  \label{fig:running-example}
\end{figure}

For example, the code in Figure~\ref{fig:running-example} is a simple program
that calculates the negation of the absolute value of the variable $\varx$.
States are pairs of labels and integers stored in $\varx$: $\stset = \labset
\times \mathbb{N}$.  Assume that the initial states are $\istset = \{ (\lab_0,
-42) \}$, which denotes that the program starts at $\lab_0$
with the variable $\varx$ of value $-42$.
Then, it executes with the following trace:
\[
  (\lab_0, -42) \trans (\lab_2, -42) \trans (\lab_3, 42) \trans (\lab_4, -42)
\]


\subsection{Abstract Interpretation}\label{sec:ai}
Abstract interpretation~\cite{abs-interp-1977, abs-interp-1992}
over-approximates the transfer function $\transfer$ as an \textit{abstract transfer
function} $\abstransfer: \absdom \rightarrow \absdom$ to get an
\textit{abstract semantics} $\abssem{\prog}$ in finite iterations as follows:
\[
    \abssem{\prog} = \underset{n \rightarrow
    \infty}{\lim}{(\abstransfer)^n(\iabselem)}\\
\]
We define a \textit{state abstraction} $\dom \galois{\alpha}{\gamma} \absdom$ as
a Galois connection between the concrete domain $\dom$ and an abstract domain
$\absdom$ with a \textit{concretization function} $\gamma$ and an
\textit{abstraction function} $\alpha$.  The initial abstract state $\iabselem
\in \absdom$ represents an abstraction of the initial state set: $\ielem
\subseteq \gamma(\iabselem)$.  The abstract transfer function $\abstransfer:
\absdom \rightarrow \absdom$ is defined as $\abstransfer(\abselem) = \abselem
\join \absstep(\abselem)$ with an \textit{abstract one-step execution}
$\absstep: \absdom \rightarrow \absdom$.  For a sound state abstraction, the
join operator and the abstract one-step execution should satisfy the following
conditions:
\begin{align}
  \forall \abselem_0, \abselem_1 \in \absdom & . \; \gamma(\abselem_0) \cup
  \gamma(\abselem_1) \subseteq \gamma(\abselem_0 \join
  \abselem_1) \label{equ:sound-join}\\
  \forall \abselem \in \absdom & . \; \step \circ \gamma(\abselem) \subseteq
  \gamma \circ \absstep(\abselem) \label{equ:sound-step}
\end{align}

A simple example abstract domain is $\absdom_\pm = \powerset{\{ -, +, 0 \}}$ with
set operators as domain operators; $-$ denotes negative integers, $+$ positive
integers, and $0$ zero.  Assume that we analyze the code in
Figure~\ref{fig:running-example} with the abstract domain and the initial abstract state $\iabselem =
\{ - \}$. Then, the analysis result is $\{ -, + \}$ because $\varx$ can
have a positive value by executing $\varx = -\varx$ but there is no
way for $\varx$ to have $0$ in this program.


\subsection{Analysis Sensitivity}\label{sec:sens}

Abstract interpretation is often defined with \textit{analysis sensitivity} to
increase the precision of static analysis.  A sensitive abstract domain
$\sabsdom: \viewset \rightarrow \absdom$ is defined with a \textit{view
abstraction} $\viewmap: \viewset \rightarrow \dom$ that provides multiple points
of views for reachable states during static analysis.  It maps a finite number
of views $\viewset$ to sets of states $\dom$. Each view $\view \in \viewset$
represents a set of states $\viewmap(\view)$ and each state is included
in a unique view: $\forall \st \in \stset. \; \st \in \viewmap(\view)
\Rightarrow \forall \view' \in \viewset. \st \in \viewmap(\view') \Rightarrow \view = \view'$.
A \textit{sensitive state
abstraction} $\dom \galois{\alpha_\viewmap}{\gamma_\viewmap} \sabsdom$ is a
Galois connection between the concrete domain $\dom$ and the sensitive abstract
domain $\sabsdom$ with the following concretization function:
\[
  \sgamma(\sabselem) = \underset{\view \in \viewset}{\bigcup}
  {\viewmap(\view) \cap \gamma \circ \sabselem(\view)}
\]

With analysis sensitivities, the abstract one-step execution $\sabsstep:
\sabsdom \rightarrow \sabsdom$ is defined as follows:
\[
  \sabsstep(\sabselem) = \lambda \view \in \viewset. \; \underset{\view' \in
  \viewset}{\bigjoin}{\viewtrans{\view'}{\view} \circ \sabselem(\view')}
\]
where $\viewtrans{\view'}{\view}: \absdom \rightarrow \absdom$ is an abstract
semantics of a \textit{view transition} from a view $\view'$ to another view
$\view$.  It should satisfy the following condition for the soundness of the
analysis:
\[
  \forall \abselem \in \absdom. \; \step(\gamma(\abselem) \cap \viewmap(\view'))
  \cap \viewmap(\view) \subseteq \gamma \circ
  \viewtrans{\view'}{\view}(\abselem)
\]

One of the most widely-used analysis sensitivity is \textit{flow sensitivity}
defined with a flow-sensitive view abstraction $\fsviewmap: \labset
\rightarrow \dom$ where:
\[
  \forall \lab\in\labset. \; \fsviewmap(\lab) = \{ \st \mid \st = (\lab, \_) \}
\]
If we apply the flow sensitivity for the above example with the initial abstract
state $[ \lab_0 \mapsto \{ -, 0, + \} ]$, the analysis result is as follows:
\[
  \begin{array}{|c||c|c|c|c|c|}\hline
    \labset & \lab_0 & \lab_1 & \lab_2 & \lab_3 & \lab_4\\\hline
    \absdom_\pm & -, 0, + & 0, + & - & 0, + & -, 0\\\hline
  \end{array}
\]


\subsection{Sealed Symbolic Execution}

We define \textit{sealed symbolic execution} by extending the transition
relation $\trans$ as a symbolic transition relation $\symbtrans$ on symbolic
states.  First, we extend concrete states $\stset$ to symbolic states
$\symbstset$ by extending values $\valset$ with \textit{sealed symbolic values}
$\symbset$.  We also define the symbolic transition relation $\symbtrans
\subseteq \symbstset \times \symbstset$. We use the notation $\symbtrans^k$
for $k$ repetition of $\symbtrans$, and write $\symbst \symbtrans \excst$ when
$\symbst$ does not have any symbolic transitions to other sealed symbolic
states.  We define the validity of sealed symbolic execution as follows:
\begin{definition}[Validity]\label{def:valid-symbtrans}
  The symbolic transition relation is \textit{valid} when the following
  condition is satisfied for any sealed symbolic states $\symbst$ and
  $\symbst'$:
  \[
    \symbst \symbtrans \symbst' \Leftrightarrow
    \forall \imap \in \imapset. \;
    \{ \st' \mid \instant{\symbst}{\imap} \trans \st' \}
    = \{ \instant{\symbst'}{\imap} \}
  \]
  where $\imapset: \symbset \rightarrow \valset$ represent \textit{instantiation
  maps} from symbolic values to concrete values, and $\instant{\symbst}{\imap}$
  denotes a state produced by replacing each symbolic value $\symb$ in
  $\symbst$ with its
  corresponding value $\imap(\symb)$ using the instantiation map $\imap \in
  \imapset$.
\end{definition}

Sealed symbolic execution is different from traditional
symbolic execution~\cite{symbolic} in that it supports only sealed symbolic
values instead of symbolic expressions and path constraints.  For example, the
following trace represents traditional symbolic execution of the running
example in Figure~\ref{fig:running-example}:
{
\small
\[
  \begin{array}{r@{~}c@{~}c@{~}c@{~}r@{~}c@{~}r}
    &&(\lab_1, \symb)[\symb \!\geq\! 0]
    &\trans& (\lab_3, \phantom{-}\symb)[\symb \!\geq\! 0]
    &\trans& (\lab_4, -\symb)[\symb \!\geq\! 0]
    \vspace*{-0.5em}\\
    &\rutrans&
    \vspace*{-0.5em}\\
    (\lab_0, \symb)[\varnothing]
    \vspace*{-0.5em}\\
    &\rdtrans&
    \vspace*{-0.5em}\\
    &&(\lab_2, \symb)[\symb \!<\! 0]
    &\trans& (\lab_3, -\symb)[\symb \!<\! 0]
    &\trans& (\lab_4, \phantom{-}\symb)[\symb \!<\! 0]\\
  \end{array}
\]
}
It first assigns a symbolic value $\symb$ to the variable $\varx$ at $\lab_0$.
For the conditional branch, it creates two symbolic states with
different path conditions $\symb \geq 0$ and $\symb < 0$ for true and false
branches, respectively.  After executing statements $\varx = \varx$ and $\varx =
-\varx$, the variable $\varx$ stores symbolic expressions $\symb$ and $-\symb$
at $\lab_3$, respectively. Similarly, $\varx$  stores $-\symb$ and $\symb$ at $\lab_4$.
However, sealed symbolic execution stops at $\lab_0$ as follows:
\[
  (\lab_0, \symb) \; \symbtrans \; \excst
\]
because the branch requires the actual value of the symbolic value $\symb$.

To define an abstract domain that contains sealed symbolic states, we define
\textit{abstract instantiation maps} $\absimapset: \symbset \rightarrow
\absvalset$ from symbolic values to abstract values.  Its concretization
function $\imapgamma: \absimapset \rightarrow \powerset{\imapset}$ is defined
with the concretization function $\valgamma: \absvalset \rightarrow
\powerset{\valset}$ for values as follows:
\[
  \imapgamma(\absimap) = \{
    \imap \mid \forall \symb \in \symbset. \;
    \imap(\symb) \in \gamma \circ \absimap(\symb)
  \}
\]
The instantiation of a given sealed symbolic state $\symbst \in \symbstset$ with
an abstract instantiation map $\absimap \in \absimapset$ is defined as follows:
\[
  \instant{\symbst}{\absimap} = \{ \instant{\symbst}{\imap} \mid \imap \in
  \imapgamma(\absimap) \}
\]
Now, we define a \textit{sealed symbolic domain} as follows:

\begin{definition}[Sealed Symbolic Domain]\label{def:symbdom}
  A \textit{sealed symbolic domain} $\symbdom: \powerset{\absimapset \times
  \symbstset}$ is defined with the concretization function
  $\symbgamma: \symbdom \rightarrow \dom$ and the sealed symbolic one-step execution
  $\symbstep: \symbdom \rightarrow \symbdom$ such that
  \begin{align}
    \symbgamma(\symbelem) &=
    \bigcup \{ \instant{\symbst}{\absimap} \mid (\absimap, \symbst) \in
    \symbelem\}\\
    \symbstep(\symbelem) &= \{ (\absimap, \symbst') \mid (\absimap, \symbst)
    \in \symbelem \wedge \symbst \symbtrans \symbst' \}
  \end{align}
\end{definition}

\subsection{Combined Domain}
We now define a \textit{combined domain} of a given sensitive abstract
domain with the sealed symbolic domain and its one-step execution.
\begin{definition}[Combined Domain]
  A \textit{combined domain} is $\combdom = \sabsdom \times \symbdom$ and its
  concretization function $\combgamma: \combdom \rightarrow \dom$ and join
  operator are defined as follows:
  \begin{align}
    \combgamma((\sabselem, \symbelem)) &= \sgamma(\sabselem) \cup
      \symbgamma(\symbelem)\\
    (\sabselem, \symbelem) \join ({\sabselem}', \symbelem') &= (\sabselem \join
      {\sabselem}', \symbelem \cup \symbelem')
  \end{align}
\end{definition}

Before defining the one-step execution for the combined domain, we introduce
\textit{analysis elements} to easily configure different types of abstract
states in the sensitive abstract domain and the sealed symbolic domain.
\begin{definition}[Analysis Elements]\label{def:aelem}
  An \textit{analysis element} $\aelem \in \aelemset = (\viewset \times \absdom)
  \uplus (\absimapset \times \symbstset)$ is either 1) a pair of a view and an
  abstract state in a sensitive abstract domain $\sabsdom$, or 2) a pair of an
  abstract instantiation map and a sealed symbolic state in a sealed symbolic
  domain $\symbdom$.  Its concretization function $\aelemgamma:
  \aelemset \rightarrow \dom$ is defined as follows:
  \[
    \aelemgamma(\aelem) = \left\{
      \begin{array}{ll}
        \viewmap(\view) \cap \gamma(\abselem) & \text{if} \; (\view, \abselem) = \aelem\\
        \instant{\symbst}{\absimap} & \text{if} \; (\absimap, \symbst) = \aelem\\
      \end{array}
    \right.
  \]
\end{definition}

\begin{figure*}[t]
  \centering
  \begin{subfigure}[t]{0.15\textwidth}
    \includegraphics[height=3.2cm]{../img/listing}
    \caption{Notations}
    \label{fig:ds-example1}
  \end{subfigure}
  \quad
  \begin{subfigure}[t]{0.23\textwidth}
    \includegraphics[height=3.2cm]{../img/path-1}
    \caption{$\varx = 0$}
    \label{fig:ds-example2}
  \end{subfigure}
  \begin{subfigure}[t]{0.28\textwidth}
    \includegraphics[height=3.2cm]{../img/path-2}
    \caption{$\varx > 0$}
    \label{fig:ds-example3}
  \end{subfigure}
  \begin{subfigure}[t]{0.28\textwidth}
    \includegraphics[height=3.2cm]{../img/path-3}
    \caption{$\varx \in \mathbb{N}$}
    \label{fig:ds-example4}
  \end{subfigure}
  \caption{Abstract interpretation using a combined domain for the running
  example with different initial values for $\varx$.}
  \label{fig:ds-examples}
\end{figure*}

Moreover, to freely convert between different kinds of analysis elements, we define two converters:
\begin{align}
  \asconverter & : (\viewset \times \absdom) \hookrightarrow
    (\absimapset \times \symbstset)\\
  \saconverter & : (\viewset \times
    \absdom) \leftarrow (\absimapset \times \symbstset)
\end{align}
While the converter $\saconverter$ is total, the other one $\asconverter$ is
\textit{partial}. Thus, it is possible to convert an analysis element
$(\view, \abselem)$ in a sensitive abstract domain to another analysis element in
a sealed symbolic domain only if the convert $\asconverter$ is defined: $(\view,
\abselem) \in \Dom(\asconverter)$.  In addition, they should convert given
analysis elements without loss of information for all $\aelem \in \aelemset$:
\[
  \asconverter(\aelem) = \aelem' \Rightarrow \left\{
  \begin{array}{l}
    \aelem = \saconverter(\aelem')\\
    \aelemgamma(\aelem) = \aelemgamma(\aelem')\\
  \end{array}
  \right.
\]

Now, we define the \textit{combined one-step execution} $\combstep: \combdom
\rightarrow \combdom$ with two converters $\asconverter$ and $\saconverter$.
It consists of two steps: 1) the \textit{\reformname} step converts
analysis elements if a new sealed symbolic execution starts or an
existing one stops, and 2) the \textit{execution} step performs execution of each
analysis element using the abstract one-step execution $\sabsstep$ in the sensitive
abstract domain and the sealed symbolic one-step execution $\symbstep$ in the sealed
symbolic domain.
\begin{definition}[Combined One-Step Execution]
A \textit{combined one-step execution} $\combstep: \combdom \rightarrow
\combdom$ is define as follows:
  \[
    \combstep(\combelem) = (\sabsstep(\sabselem), \symbstep(\symbelem))
  \]
where $(\sabselem, \symbelem) = \reform(\combelem)$.
\end{definition}

From a given combined state $\combelem$, the $\reform$ function makes analysis elements
and converts them if a new sealed symbolic execution
begins or an existing sealed symbolic execution terminates.
Specifically, for an analysis element $(\view, \abselem)$ in the sensitive abstract domain,
if the converter $\asconverter$ is defined for it, $\reform$ introduces a new sealed symbolic execution
by converting the analysis element to its corresponding one $(\absimap, \symbst) =
\asconverter((\view, \abselem))$ in the sealed symbolic domain.
On the other hand, for an analysis element $(\absimap, \symbst)$ in the sealed symbolic domain,
if it does not have any sealed symbolic states to transit to, $\symbst \symbtrans \excst$,
the sealed symbolic execution for $(\absimap, \symbst)$ terminates;
It converts the analysis element to its corresponding one $(\view, \abselem) =
\saconverter((\absimap, \symbst))$ in the sensitive abstract domain and
merges the current abstract state stored in the view $\view$ with $\abselem$.

To formally define the $\reform$ function, we first define a $\areform$ function
for analysis elements using two converters.
\begin{definition}[$\areform$]\label{def:areform}
  The function $\areform: \aelemset \rightarrow \aelemset$ for analysis elements
  is defined as follows:
  \[
    \areform(\aelem) = \left\{
      \begin{array}{ll}
        \asconverter(\aelem)
        & \text{if} \; \aelem = (\view, \abselem) \wedge \aelem \in
        \Dom(\asconverter)\\
        \saconverter(\aelem)
        & \text{if} \; \aelem = (\absimap, \symbst) \wedge \symbst \symbtrans
        \bot\\
        \aelem
        & \text{Otherwise}
      \end{array}
    \right.
  \]
\end{definition}
\begin{definition}[$\reform$]\label{def:reform}
  The \reformname function $\reform: \combdom \rightarrow \combdom$ for combined
  states is defined as follows:
  \[
    \reform((\sabselem, \symbelem)) = \left(
      \lambda \view. \bigjoin \{ \abselem \!\mid\! (\view, \abselem) \in E \},
      E \cap (\absimapset \times \symbstset)
    \right)
  \]
  where
  \[
    E = \dot{\areform}(\{ (\view, \sabselem(\view)) \mid \view \in \viewset \} \cup \symbelem)
  \]
and the dot notation $\dot{f}$ denotes the element-wise extended function of a
function $f$.
\end{definition}


\subsection{Examples}
Now, we show examples of abstract interpretation with a combined domain.
Figure~\ref{fig:ds-examples} depicts the flow of analysis for the running
example in Figure~\ref{fig:running-example} with three different initial sets of
values for the variable $\varx$.  In this example, we use the abstract domain
$\{ -, 0, + \}$ for integers stored in $\varx$ as introduced in
Section~\ref{sec:ai}, and the \textit{flow sensitivity} that utilizes the
labels of states as their views as introduced in Section~\ref{sec:sens}.
For brevity, we use concatenation of abstract values so that
$-0$ denotes the set $\{ -, 0 \}$.

Figure~\ref{fig:ds-examples}(a) presents notations used in each graph. A solid
box denotes an analysis element that is a pair of a label $\lab$ and an abstract
state $\abselem$.  A pair enclosed by angle brackets denotes an analysis
element that is a pair of an abstract instantiation map $\absimap$ and a sealed
symbolic state $\symbst$.  In fact, the sealed symbolic state part (right) of
each pair in graphs contains only the value of the variable of $\varx$ without
its label.  For brevity, we represent its label by locating it next to
a node with its label.  A solid line is a view transition
$\viewtrans{\lab}{\lab'}$ from a label $\lab$ to another one $\lab'$.  A dotted
line is a sealed symbolic transition $\symbtrans$.  Three solid lines with
circled labels denote two converters $\saconverter$, $\asconverter$ and the join
operator $\join$.

Figure~\ref{fig:ds-examples}(b) shows the analysis with the combined domain when
the initial value of $\varx$ is $0$.  First, in the \reformname step,
the converter $\asconverter$ converts the analysis element $(\lab_0, 0)$ to
another analysis element $\langle \varnothing, 0 \rangle$ with the label
$\lab_0$.  It does not introduce any sealed symbolic values because
the value represents only a single value.  Until the end of the program, the
sealed symbolic execution from $\langle \varnothing, 0 \rangle$ successfully
continues.  Because there is no more possible symbolic transition for the
symbolic state $\langle \varnothing, 0 \rangle$ with the label $\lab_4$,
it is converted to $(\lab_4, 0)$ via the converter $\saconverter$.

Instead of a single value, assume that the initial value of $\varx$ is one of
any positive integers.  Figure~\ref{fig:ds-examples}(c) describes the analysis
flow for the case.  The initial abstract value at the label $\lab_0$ is
$+$ and it is impossible to convert it to any sealed symbolic values because the
next program statement requires the actual value stored in the variable $\varx$
for the branch condition $\varx \geq 0$.  Thus, it performs view transition
$\viewtrans{\lab_0}{\lab_1}$ from the label $\lab_0$ to another one $\lab_1$ for
the abstract value $+$ and the result is also $+$.  Now, the analysis element
$(\lab_1, +)$ can be converted to $\langle \symb \mapsto +, \symb \rangle$
with the label $\lab_1$.  This sealed symbolic execution step terminates in the
label $\lab_3$ because the next statement is $\varx = -\varx$ and the negation
operator requires the actual value of $\varx$.  It is converted to $(\lab_3, +)$ via $\saconverter$,
performs the view transition, and results in $(\lab_4, -)$.

For the last case, we assume that all integers are possible for the initial
value of the variable $\varx$ as described in Figure~\ref{fig:ds-examples}(d).
While it reaches the false branch in the label $\lab_2$ unlike previous cases,
it cannot perform dynamic shortcuts because the statement in the false
branch is $\varx = -\varx$, which requires the actual value of $\varx$.
At the label $\lab_3$, there are two analysis
elements: 1) $(\lab_3, +)$ introduced by the view transition from the label $\lab_2$
with $-$, and 2) $\langle \symb \mapsto 0+, \symb \rangle$ with $\lab_3$
introduced by sealed symbolic execution started at $\lab_1$.  Since it
is not possible to perform sealed symbolic execution for both elements, the
second one is converted to $(\lab_3, 0+)$ and merged with $+$ at $\lab_3$ via the
join operator $\join$.  Finally, the view transition
$\viewtrans{\lab_3}{\lab_4}$ from $\lab_3$ to $\lab_4$ is performed to the
merged abstract state $0+$ and the result is $-0$.

\subsection{Soundness and Termination}
The converter $\asconverter$ and the sealed symbolic transition $\symbtrans$ are
keys to configure the introduction and termination of sealed symbolic
execution.  To ensure the \textit{soundness} and \textit{termination} of an
abstract interpretation defined with a combined domain of a sensitive abstract
domain and a sealed symbolic domain, the following conditions should hold.

\begin{theorem}[Soundness and Termination]\label{theorem:shortcut}
An abstract interpretation with dynamic shortcuts is \textbf{sound} and
\textbf{terminates} in a finite time if:
  \begin{itemize}
    \item the abstract transfer function $\abstransfer$ is sound,
    \item the sensitive abstract domain $\sabsdom$ has a finite height,
    \item the sealed symbolic transition $\symbtrans$ is valid, and
    \item there exists $N < \infty$ such that
      \[
        \forall \aelem \in \aelemset. \; \asconverter(\aelem) = (\absimap,
        \symbst) \Rightarrow \symbst
        \symbtrans^k \excst \wedge 1 < k \leq N
      \]
  \end{itemize}
\end{theorem}

To formally prove Theorem~\ref{theorem:shortcut}, we assume that its all
conditions are hold and rephrase the \textit{soundness} as
Theorem~\ref{theorem:soundness} and \textit{termination} as
Theorem~\ref{theorem:termination}.

\subsubsection{Soundness}

\begin{theorem}[Soundness]\label{theorem:soundness}
  The abstract interpretation using the combined domain $\combdom$ is
  \textbf{sound} if
  \begin{equation}\label{equ:sound-join}
    \forall \combelem_0, \combelem_1 \in \combdom. \; \combgamma(\combelem_0) \cup
    \combgamma(\combelem_1) \subseteq \combgamma(\combelem_0 \join \combelem_1)
  \end{equation}
  \begin{equation}\label{equ:sound-combstep}
    \forall \combelem \in \combdom. \; \step \circ \combgamma(\combelem) \subseteq
    \combgamma \circ \combstep(\combelem)\\
  \end{equation}
\end{theorem}
\begin{proof}
  First, we prove that the abstract transfer function $\combtransfer: \combdom
  \rightarrow \combdom$ defined as $\combtransfer(\combelem) = \combelem \join
  \combstep(\combelem)$ is sound
  \[
    \begin{array}{rcll}
      \transfer \circ \combgamma(\combelem)
      &=& \combgamma(\combelem) \cup \step \circ \combgamma(\combelem)\\
      &\subseteq& \combgamma(\combelem) \cup \combgamma \circ \combstep(\combelem)
      & (\because \; \text{condition~(\ref{equ:sound-combstep})})\\
      &\subseteq& \combgamma(\combelem \join \combstep(\combelem))
      & (\because \; \text{condition~(\ref{equ:sound-join})})\\
      &=& \combgamma \circ \combtransfer(\combelem)\\
    \end{array}
  \]
  Then, the abstract semantics $\combsem{\prog} = \underset{n \rightarrow
  \infty}{\lim}{(\combtransfer)^n(\icombelem)} $ is also sound because it is
  defined with a sound abstract transfer function $\combtransfer$ using the
  combined one-step execution $\combstep$.
\end{proof}

Now, we should show that two conditions about the soundness of the join
operator (\ref{equ:sound-join}) and the soundness of the combined one-step
execution (\ref{equ:sound-combstep}) in Theorem~\ref{theorem:soundness} hold.

First, we prove the soundness of the join operator (\ref{equ:sound-join}) in
Lemma~\ref{lemma:sound-join}.
\begin{lemma}[Soundness of $\join$]\label{lemma:sound-join}
  \[
    \forall \combelem_0, \combelem_1 \in \combdom. \; \combgamma(\combelem_0) \cup
    \combgamma(\combelem_1) \subseteq \combgamma(\combelem_0 \join \combelem_1)
  \]
\end{lemma}
\begin{proof}
  \[
    \begin{array}{cl}
      \multicolumn{2}{l}{
        \combgamma((\sabselem, \symbelem)) \cup \combgamma((\sabselem', \symbelem'))
      }\\
      =& \sgamma(\sabselem) \cup \symbgamma(\symbelem)
      \cup \sgamma(\sabselem') \cup \symbgamma(\symbelem')\\
      =& (\sgamma(\sabselem)\cup \sgamma(\sabselem'))
      \cup (\symbgamma(\symbelem) \cup \symbgamma(\symbelem'))\\
      \subseteq& \sgamma(\sabselem \join \sabselem')
      \cup (\symbgamma(\symbelem) \cup \symbgamma(\symbelem'))\\
      & \multicolumn{1}{r}{(\because \; \sabsdom \; \text{is sound})}\\
      =& \sgamma(\sabselem \join \sabselem')
      \cup \symbgamma(\symbelem \cup \symbelem')\\
      =& \combgamma((\sabselem \join \sabselem', \symbelem \cup \symbelem'))\\
      =& \combgamma((\sabselem, \symbelem) \join (\sabselem', \symbelem'))\\
    \end{array}
  \]
\end{proof}

For the condition (\ref{equ:sound-combstep}), we first prove two properties of the
$\reform$ function in Lemma~\ref{lemma:reform}.  Using the properties, we prove
the soundness of the sealed symbolic one-step execution in
Lemma~\ref{lemma:sound-symbstep}.  Finally, we prove the soundness of the
combined one-step execution (\ref{equ:sound-combstep}) in
Lemma~\ref{lemma:sound-combstep}.

\begin{lemma}[Properties of $\reform$]\label{lemma:reform}
  For a given combined state $\combelem \in \combdom$, the $\reform$ function
  satisfies the following two properties:
  \begin{itemize}
    \item $\combgamma(\combelem) \subseteq \combgamma \circ \reform(\combelem)$
    \item $\forall (\absimap, \symbst) \in \symbelem. \; \exists \symbst' \in
      \symbstset.  \; \text{s.t.} \; \symbst \symbtrans \symbst'$
  \end{itemize}
  where $(\sabselem, \symbelem) = \reform(\combelem)$
\end{lemma}
\begin{proof}
  \[
    \fbox{$\combgamma(\combelem) \subseteq \combgamma \circ \reform(\combelem)$}
  \]
  \[
    \begin{array}{cl}
      \multicolumn{2}{l}{\combgamma((\sabselem, \symbelem))}\\
      =& \sgamma(\sabselem) \cup \symbgamma(\symbelem)\\

      =& \left( \underset{\view \in \viewset}{\bigcup} {\viewmap(\view) \cap
      \gamma \circ \sabselem(\view)} \right) \cup \left( \underset{(\absimap,
      \symbst) \in \symbelem}{\bigcup} \instant{\symbst}{\absimap} \right) \\

      =& \left( \underset{\view \in \viewset}{\bigcup} \aelemgamma((\view,
      \sabselem(\view))) \right) \cup \left( \underset{(\absimap, \symbst) \in
      \symbelem}{\bigcup} \aelemgamma((\absimap, \symbst)) \right) \\

      =& \dot\aelemgamma(\{ (\view, \sabselem(\view)) \mid \view \in \viewset \}
      \cup \symbelem)\\

      =& \dot\aelemgamma(\dot{\areform}(\{ (\view, \sabselem(\view)) \mid \view
      \in \viewset \} \cup \symbelem))\\

       & \multicolumn{1}{r}{\because \; (\text{Trivially,} \; \forall \aelem \in
       \aelemset.  \; \aelemgamma(\aelem) = \aelemgamma \circ \areform(\aelem))}\\

      =& \dot\aelemgamma(E)\\
       & \multicolumn{1}{r}{\because \; (\text{See the definition of $E$ in
       Definition~\ref{def:reform}})}\\
    \end{array}
  \]
  \[
    \begin{array}{cl}
      =& \left( \underset{(\view, \abselem) \in E}{\bigcup} \aelemgamma((\view,
      \abselem)) \right) \cup \left( \underset{(\absimap, \symbst) \in
      E}{\bigcup} \aelemgamma((\absimap, \symbst)) \right)\\

      =& \left( \underset{(\view, \abselem) \in E}{\bigcup} \viewmap(\view) \cap
      \gamma(\abselem) \right) \cup \left( \underset{(\absimap, \symbst) \in
      E}{\bigcup} \instant{\symbst}{\absimap} \right)\\

      =& \left( \underset{\view \in \viewset}{\bigcup} { \underset{(\view,
      \abselem) \in E}{\bigcup} \viewmap(\view) \cap \gamma(\abselem) } \right)
      \cup \left( \underset{(\absimap, \symbst) \in E}{\bigcup}
      \instant{\symbst}{\absimap} \right)\\

      =& \left( \underset{\view \in \viewset}{\bigcup} {\viewmap(\view) \cap
        \left(\underset{(\view, \abselem) \in
      E}{\bigcup}{\gamma(\abselem)}\right)} \right) \cup \left(
      \underset{(\absimap, \symbst) \in E}{\bigcup} \instant{\symbst}{\absimap}
      \right)\\

      \subseteq& \left( \underset{\view \in \viewset}{\bigcup} {\viewmap(\view)
        \cap \gamma\left( \underset{(\view, \abselem) \in E}{\bigjoin}\abselem
      \right)} \right) \cup \left( \underset{(\absimap, \symbst) \in E}{\bigcup}
      \instant{\symbst}{\absimap} \right)\\

      =& \sgamma\left( \lambda \view. \underset{(\view, \abselem) \in
      E}{\bigjoin}\abselem \right) \cup \symbgamma(E \cap (\absimapset \times
      \symbstset))\\

      =& \combgamma\left( \lambda \view. \underset{(\view, \abselem) \in
      E}{\bigjoin}\abselem, E \cap (\absimapset \times \symbstset) \right)\\

      =& \combgamma \circ \reform((\sabselem, \symbelem))
    \end{array}
  \]
  \[\]
  \[
    \fbox{$\forall (\absimap, \symbst) \in \symbelem. \; \exists \symbst' \in
    \symbstset.  \; \text{s.t.} \; \symbst \symbtrans \symbst'$}
  \]

  For a given $(\absimap, \symbst) \in \symbelem$, there exists an analysis
  element $\aelem \in \aelemset$ such that $\areform(\aelem) = (\absimap,
  \symbst)$.  According to the definition of $\areform$ in
  Definition~\ref{def:areform}, there are two possible cases: $\aelem =
  (\absimap, \symbst) \wedge \exists \symbst' \in \symbstset. \; \text{s.t} \;
  \symbst \symbtrans \symbst'$ or $\aelem = (\view, \abselem) \wedge \aelem \in
  \Dom(\asconverter)$. We separately consider those two cases:
  \begin{itemize}
    \item $\aelem = (\absimap, \symbst) \wedge \exists \symbst' \in \symbstset.
      \; \text{s.t} \; \symbst \symbtrans \symbst'$\\
        By definition, $\exists \symbst' \in \symbstset.  \; \text{s.t} \;
        \symbst \symbtrans \symbst'$
    \item $\aelem = (\view, \abselem) \wedge \aelem \in \Dom(\asconverter)$\\
      By the condition~\ref{equ:asc-cond} in the Theorem~\ref{theorem:shortcut},\\
      $\exists k > 1. \symbst \symbtrans^k \excst$.  Thus, $\exists \symbst' \in
      \symbstset.  \; \text{s.t} \; \symbst \symbtrans \symbst'$
  \end{itemize}
\end{proof}

\begin{lemma}[Soundness of $\symbstep$]\label{lemma:sound-symbstep}
  The sealed symbolic one-step execution $\symbstep$ is sound:
  \[
    \step \circ \symbgamma(\symbelem) \subseteq
    \symbgamma \circ \symbstep(\symbelem)\\
  \]
\end{lemma}
\begin{proof}
  \[
    \begin{array}{cl}
      \multicolumn{2}{l}{\step \circ \symbgamma(\symbelem)}\\
      =& \step(\bigcup \{ \instant{\symbst}{\absimap} \mid (\absimap, \symbst) \in
      \symbelem\})\\
      =& \{ \st' \mid (\absimap, \symbst) \in \symbelem \wedge \st \in
      \instant{\symbst}{\absimap} \wedge \st \trans \st'\})\\
      =& \{ \st' \mid (\absimap, \symbst) \in \symbelem \wedge \imap \in
      \imapgamma(\absimap) \wedge \instant{\symbst}{\imap} \trans \st'\})\\
      =& \{ \st' \mid (\absimap, \symbst) \in \symbelem \wedge \imap \in
      \imapgamma(\absimap) \wedge \instant{\symbst}{\imap} \trans \st'\\
       & \phantom{\{ \st' \mid (\absimap, \symbst) \in \symbelem \wedge \imap \in
      \imapgamma(\absimap)} \wedge \symbst \symbtrans \symbst' \})\\
       & \multicolumn{1}{r}{(\because \; \text{Second property in
       Lemma~ref{lemma:reform}})}\\
      =& \{ \instant{\symbst'}{\imap} \mid (\absimap, \symbst)
      \in \symbelem \wedge \imap \in \imapgamma(\absimap) \wedge \symbst
      \symbtrans \symbst'\}\\
      & \multicolumn{1}{r}{(\because \; \text{Validity of} \; \symbtrans)}\\
      =& \bigcup \{ \instant{\symbst'}{\absimap} \mid (\absimap, \symbst)
      \in \symbelem \wedge \symbst \symbtrans \symbst' \}\\
      =& \symbgamma(\{ (\absimap, \symbst') \mid (\absimap, \symbst)
      \in \symbelem \wedge \symbst \symbtrans \symbst' \})\\
      =& \symbgamma \circ \symbstep(\symbelem)\\
    \end{array}
  \]
\end{proof}

\begin{lemma}[Soundness of $\combstep$]\label{lemma:sound-combstep}
  The combined one-step execution $\combstep$ is sound:
  \[
    \forall \combelem \in \combdom. \; \step \circ \combgamma(\combelem) \subseteq
    \combgamma \circ \combstep(\combelem)\\
  \]
\end{lemma}
\begin{proof}
  \[
    \begin{array}{cl}
      \multicolumn{2}{l}{
        \step \circ \combgamma(\combelem)
      }\\
      \subseteq& \step \circ \combgamma((\sabselem, \symbelem))\\
       & \multicolumn{1}{r}{(\because \; \text{First property in
       Lemma~\ref{lemma:reform}}}\\
       & \multicolumn{1}{r}{\text{where} \; (\sabselem, \symbelem)
       = \reform(\combelem).)}\\

      =& \step(\sgamma(\sabselem) \cup \symbgamma(\symbelem))\\
      =& \step(\sgamma(\sabselem)) \cup \step(\symbgamma(\symbelem))\\
      \subseteq& \sgamma \circ \sabsstep(\sabselem) \cup \step(\symbgamma(\symbelem))\\
      & \multicolumn{1}{r}{(\because \; \sabsstep \; \text{is sound.})}\\
      \subseteq& \sgamma \circ \sabsstep(\sabselem) \cup \symbgamma \circ
      \symbstep((\symbelem))\\
               & \multicolumn{1}{r}{(\because \; \text{and
               Lemma~\ref{lemma:sound-symbstep}})}\\
      =& \combgamma((\sabsstep(\sabselem), \symbstep(\symbelem)))\\
      =& \combgamma \circ \combstep(\combelem)\\
    \end{array}
  \]
\end{proof}


\subsection{Termination}

Before proving the termination of the abstract interpretation using the combined
domain $\combdom$, we define several notations. The initial abstract state
$\icombelem = (\isabselem, \varnothing)$ is pair of the initial abstract state of
the sensitive abstract domain $\sabsdom$ and an empty set. For each iteration $i
\geq 0$, we define the $i$-th result of abstract interpretation
$\combtransfer^i(\icombelem) = \combelem^i = (\sabselem^i, \symbelem^i)$ and the
\textit{difference set} $\diffset_i = \symbelem^{i+1} \setminus \symbelem^i$.
For simplicity, we define $\diffset_i$ as $\varnothing$ for $i < 0$.  Moreover, we
define a lifted version of sealed symbolic relation $\liftsymbtrans \subseteq
(\absimapset \times \symbstset) \times (\absimapset \times \symbstset)$ as
follows:
\[
  (\absimap, \symbst) \liftsymbtrans (\absimap, \symbst') \Leftrightarrow
  \symbst \symbtrans \symbst'
\]
Using the lifted relation, we define the \textit{time to live (TTL)} function of
symbolic states $\ttl_i: \diffset_i \rightarrow \numset$ for each iteration $i
\geq 0$ as follows:
\begin{definition}[TTL Function]
  \[
    \begin{array}{c}
      \ttl_i(\symbaelem) = \left \{
      \begin{array}{l}
        N - 1 \;\; ( \text{if} \; D = \varnothing)\\
        \text{min}(\dot\ttl_{i-1}(D)) - 1
        \;\; ( \text{otherwise})
      \end{array}
      \right. \\
      \\
      \text{where} \; D =
      \{\symbaelem' \in \diffset_{i-1} \mid \symbaelem' \liftsymbtrans \symbaelem\}
    \end{array}
  \]
\end{definition}

Based on the notations, we formally prove the termination property as folows:
\begin{theorem}[Termination]\label{theorem:termination}
  The abstract interpretation using the combined domain $\combdom$
  \textbf{terminates} in a finite time if
  \begin{equation}\label{equ:termination-sai}
    \exists n. \; \forall m \geq n. \; \sabselem^m = \sabselem^n
  \end{equation}
  \begin{equation}\label{equ:bounded-ttl}
    \forall i \geq 0. \; \forall \symbaelem \in \diffset_i. \;
    0 < \ttl_i(\symbaelem) < N
  \end{equation}
  \begin{equation}\label{equ:dec-ttl}
    \begin{array}{c}
      \forall i > 0. \; \sabselem^{i-1} = \sabselem^i \Rightarrow\\
      \sup(\dot \ttl_i(\diffset_i)) \leq \sup(\dot \ttl_{i-1}(\diffset_{i-1})) - 1
    \end{array}
  \end{equation}
\end{theorem}

\begin{proof}
  By the condition (\ref{equ:termination-sai}), there exists $n \in \numset$
  such that $\sabselem^m = \sabselem^n$ for all $m \geq n$.  By the condition
  (\ref{equ:bounded-ttl}), the TTL of each symbolic state in $\diffset_n$ is
  bounded by $N$:
  \[
    \sup(\dot \ttl_n(\diffset_n)) < N
  \].
  Then, the upper bound of TTL for symbolic states in each difference set after
  the $n-$th iteration is decreased by the condition (\ref{equ:dec-ttl}):
  \[
    \forall i > 0. \; \sup(\dot \ttl_{n+i}(\diffset_{n+i})) \leq \sup(\dot
    \ttl_{n+i-1}(\diffset_{n+i-1})) - 1
  \].
  which implies that
  \[
    \sup(\dot \ttl_{n+i}(\diffset_{n+i})) \leq \sup(\dot \ttl_n(\diffset_n)) - i < N - i
  \]
  Therefore, for $j \geq N$,
  \[
    \sup(\dot \ttl_{n+j}(\diffset_{n+j})) < N - j \leq 0
  \]
  Notice that again by the condition (\ref{equ:bounded-ttl}),
  \[
    inf(\dot \ttl_{n+j}(\diffset_{n+j})) > 0
  \]
  meaning that
  \[
    inf(\dot \ttl_{n+j}(\diffset_{n+j})) > \sup(\dot \ttl_{n+j}(\diffset_{n+j}))
  \]
  which implies $\diffset_{n+j} = \varnothing$ and $\symbelem^{n+j+1} =
  \symbelem^{n+j}$.
  Therefore, for all $m \geq n + N$,
  \[
    \sabselem^m = \sabselem^{n+N} \wedge \symbelem^m = \symbelem^{n+N}
  \]
  and
  \[
    \combelem^m = \combelem^{n+N}
  \]
  which means the abstract interpretation using the combined domain
  $\combdom$ terminates in $n+N$ iterations.
\end{proof}

Now, we should show that three conditions about the termination of the sensitive
abstract interpretation (\ref{equ:termination-sai}), the bound of TTL for
symbolic states in difference sets (\ref{equ:bounded-ttl}), and the decrease of
their upper bounds (\ref{equ:dec-ttl}) in Theorem~\ref{theorem:termination}
hold.

First, we prove the termination of the sensitive abstract interpretation
(\ref{equ:termination-sai}) in Lemma~\ref{lemma:termination-sai}.

\begin{lemma}[Termination of Sensitive Abstract Interpretation]\label{lemma:sabs-term}
\label{lemma:termination-sai}
  \[
    \exists n. \; \forall m \geq n. \;
    \sabselem^m = \sabselem^n
  \]
\end{lemma}

\begin{proof}
Note that for all $\sabselem, \sabselem' \in \sabsdom$ that satisfies
$\combtransfer((\sabselem, \_)) = (\sabselem', \_)$,
\[
  \begin{array}{rcl}
  \combtransfer((\sabselem, \_))
  &=& (\sabselem, \_) \join \combstep((\sabselem, \_))\\
  &=& (\sabselem, \_) \join (\_, \_) = (\sabselem \join \_, \_)\\
  &=& (\sabselem', \_)
  \end{array}
\]
which implies $\sabselem \order \sabselem'$.  Since
$\combtransfer((\sabselem^i, \_)) = (\sabselem^{i+1}, \_), \sabselem^i \order
\sabselem^{i+1}$ holds for all $i \geq 0$.  Then, $\sabselem^0 \order \sabselem^1
\order \sabselem^2 \cdots$ is an ascending chain.  Since the height of the
sensitive abstract domain $\sabsdom$ is finite, the ascending chain condition is
also hold. Therefore, there exists n such that for all $m \geq n, \sabselem^m =
\sabselem^n$.
\end{proof}

Then, we prove two remaining conditions (\ref{equ:bounded-ttl}) and
(\ref{equ:dec-ttl}).  We first prove two properties of difference sets in
Lemma~\ref{lemma:diffset_prop} and Corollary~\ref{corollary:only-from-diffset},
and a property of TTL in Lemma~\ref{lemma:prop-ttl}.  Using them, we prove the
bound of TTL for symbolic states in difference sets (\ref{equ:bounded-ttl}) in
Corollary~\ref{corollary:bounded-ttl} and the decrease of their upper bounds
(\ref{equ:dec-ttl}) in Lemma~\ref{lemma:dec-ttl}.

\begin{lemma}\label{lemma:diffset_prop}
  \[
    \begin{array}{c}
      \forall i \geq 0. \; \forall \symbaelem \in \diffset_i. \\
      \exists \view. \; \asconverter((\view,\sabselem^i(\view))) \liftsymbtrans \symbaelem 
      \lor \exists \symbaelem' \in \diffset_{i-1} . \; \symbaelem' \liftsymbtrans \symbaelem
    \end{array}
  \]
\end{lemma}
\begin{proof}
  Let $i \in \numset$ and $\symbaelem \in \diffset_i = \symbelem^{i+1} \setminus \symbelem^i$ given.
  By definition,
  \[
    \symbelem^{i+1} = \symbelem^i \cup \symbstep({\symbelem^i}')
  \]
  where
  \[
    (\_, {\symbelem^i}') = \reform(\sabselem^i, \symbelem^i)
  \]
  Note that $\symbaelem \in \symbstep({\symbelem^i}')$,
  and by definition of $\symbstep$, there exists some
  $\symbaelem' \in {\symbelem^i}'$ that satisfies $\symbaelem' \liftsymbtrans \symbaelem$.
  Now, by definition of $\reform$,
  \[
    {\symbelem^i}' =
    \dot{\areform}(\{ (\view, \sabselem^i(\view)) \mid \view \in \viewset \} \cup \symbelem^i)
    \cap (\absimapset \times \symbstset)
  \]
  This means there exists
  $\aelem \in \{ (\view, \sabselem^i(\view)) \mid \view \in \viewset \} \cup \symbelem^i$
  that satisfies $\areform(\aelem) = \symbaelem'$. We have two possible cases for $\aelem$.
  \begin{itemize}
 
  \item $\aelem \in \{ (\view, \sabselem^i(\view)) \mid \view \in \viewset \}$

  In this case, $\areform(\aelem) = \asconverter(\aelem) = \symbaelem'$
  and the left condition for conclusion is satisfied.
  
  \item $\aelem \in \symbelem^i$

  In this case, $\areform(\aelem) = \aelem = \symbaelem'$.
  Now, let's assume that $\aelem \in \symbelem^{i-1}$.
  In that case, $\aelem$ would be preserved after reform step, that is,
  $\aelem \in {\symbelem^{i-1}}'$. Then, by definition of $\symbstep$,
  $\symbaelem \in \symbstep({\symbelem^{i-1}}') \subseteq \symbelem^i$
  whcih contradicts to the fact that $\symbaelem \in \diffset_i$.
  Therefore, $\aelem \notin \symbelem^{i-1}$, that is,
  $\aelem \in \symbelem^i \setminus \symbelem^{i-1} = \diffset_i$,
  and the right condition for conclusion is satisfied.
  \end{itemize}
\end{proof}

\begin{corollary}\label{corollary:only-from-diffset}
  \[
    \begin{array}{c}
      \forall i > 0. \; \sabselem^{i-1} = \sabselem^i \Rightarrow
      \forall \symbaelem \in \diffset_i. \\
      \exists \symbaelem' \in \diffset_{i-1} . \; \symbaelem' \liftsymbtrans \symbaelem
    \end{array}
  \]
\end{corollary}
\begin{proof}
  The proof goes same as the previous lemma, until the point where we divide
  the case for $\aelem$. Let's assume that the first case holds, that is,
  \[
    \aelem \in \{ (\view, \sabselem^i(\view)) \mid \view \in \viewset \}
  \]
  Since $\sabselem^{i-1} = \sabselem^i$,
  \[
    \aelem \in \{ (\view, \sabselem^{i-1}(\view)) \mid \view \in \viewset \}
  \]
  In that case, $\aelem$ would be transformed after reform step, that is,
  $\asconverter(\aelem) = \symbaelem' \in {\symbelem^{i-1}}'$.
  Then, by definition of $\symbstep$,
  $\symbaelem \in \symbstep({\symbelem^{i-1}}') \subseteq \symbelem^i$
  whcih contradicts to the fact that $\symbaelem \in \diffset_i$.
  Therefore, only second case holds and the right conclusion in previus lemma is satisfied.
\end{proof}

\begin{lemma}[Property of TTL]\label{lemma:prop-ttl}
  \[
    \begin{array}{c}
      \forall i \geq 0. \; \forall \symbaelem \in \diffset_i.
      \ttl_i(\symbaelem) = k \Rightarrow \\
      k < N \wedge
      \exists (\view, \abselem). \; (\asconverter((\view,\abselem))
      \liftsymbtrans^{(N - k)} \symbaelem) \\
    \end{array}
  \]
\end{lemma}
\begin{proof}
  We prove by induction on $i$.
  Let $\symbaelem \in \diffset_i$.
  \begin{itemize}
  \item If $i = 0$, $\ttl_0(\symbaelem) = N - 1 < N$
  and since only left conclusion of lemma~\ref{lemma:diffset_prop} can hold,
  there eixsts view $\view$ s.t.
  $\asconverter(\view, \sabselem^0(\view)) \liftsymbtrans^1 \symbaelem$.
  \item If $i > 0$, we have two cases for $D = 
    \{\symbaelem' \in \diffset_{i-1} \mid \symbaelem' \liftsymbtrans \symbaelem\}$.
  If $D = \varnothing$, the argument is similar as $i = 0$ case.
  Otherwise, let $\symbaelem' = \underset{\x \in D}{argmin}{\ttl_{i-1}(x)}$.
  
  By induction hypothesis, we have
  \[
    k' = \ttl_{i-1}(\symbaelem') < N
  \]
  and there eixsts $(\view, \abselem)$ such that
  \[
    \symbaelem'' = \asconverter((\view,\abselem)) \liftsymbtrans^{(N - k')} \symbaelem'.
  \]
  By definition of $\ttl_i$,
  $\ttl_i(\symbaelem) = \ttl_i(\symbaelem') - 1$, and $k = k' - 1$.
  Then,
  \[
    k = k' - 1 < N - 1 < N
  \]
  and
  $\symbaelem'' \liftsymbtrans^{(N - k - 1)} \symbaelem'$ with
  $\symbaelem' \liftsymbtrans \symbaelem$ implies that
  \[
    \symbaelem'' \liftsymbtrans^{(N - k)} \symbaelem.
  \]
  \end{itemize}
\end{proof}
\begin{corollary}\label{corollary:bounded-ttl}
  \[
    \forall i \geq 0. \; \forall \symbaelem \in \diffset_i. \;
    0 < \ttl_i(\symbaelem) < N
  \]
\end{corollary}
\begin{proof}
We already proved $k = \ttl_i(\symbaelem) < N$.
Now, let's assume that $k \leq 0$.
By previous lemma, there exists $(\view, \abselem)$ such that
\[
  (\asconverter((\view,\abselem)) \liftsymbtrans^{(N - k)} \symbaelem)
\]
Since $N - k \geq N$, this implies that there exists $\symbaelem'$ such that
\[
  (\asconverter((\view,\abselem)) \liftsymbtrans^N \symbaelem')
\]
However, this contradicts to the condition~(\ref{equ:asc-cond}) of $\asconverter$ that says
if $(\view,\abselem)$ is in domain of $\asconverter$,
the number of possible $\symbtrans$ from state of $\asconverter((\view,\abselem))$
is at most $N - 1$.
Therefore, $k > 0$.
\end{proof}

\begin{lemma}\label{lemma:dec-ttl}
  \[
    \begin{array}{c}
      \forall i > 0. \; \sabselem^{i-1} = \sabselem^i \Rightarrow \\
      \sup(\dot \ttl_i(\diffset_i)) \leq \sup(\dot \ttl_{i-1}(\diffset_{i-1})) - 1
    \end{array}
  \]
\end{lemma}
\begin{proof}
  Let $\symbaelem \in \diffset_i$.
  By Corollary~\ref{corollary:only-from-diffset}, the set
  \[
    D = \{\symbaelem' \in \diffset_{i-1} \mid \symbaelem' \liftsymbtrans \symbaelem\}
  \]
  is non-empty, and for some $\symbaelem' \in \diffset_{i-1}$,
  \[
    \ttl_i(\symbaelem) = \ttl_{i-1}(\symbaelem') - 1 \leq \sup(\dot \ttl_{i-1}(\diffset_{i-1})) - 1
  \]
  Since it holds for every $\symbaelem \in \diffset_i$,
  \[
    \sup(\dot \ttl_i(\diffset_i)) \leq \sup(\dot \ttl_{i-1}(\diffset_{i-1})) - 1
  \]
\end{proof}

\section{Dynamic Shortcuts for JavaScript}\label{sec:javascript}
In this section, we introduce the core language of JavaScript that supports
first-class functions, open objects, and first-class property names, and define
{\sealed} execution of the core language for dynamic shortcuts.
Due to the space limitation, we present the main design of the
language in this paper and refer the interested readers to a companion report~\cite{report}.

\subsection{Core Language of JavaScript}

\[
  \begin{array}{ll@{~}c@{~}l}
    \text{Programs} & \prog &::=& (\lab: \inst)^*\\

    \text{Labels} & \lab &\in& \labset\\

    \text{Instructions} & \inst &::=&
    \refer = \expr \mid
    \refer = \kwobj \mid
    \refer = \expr ( \expr ) \mid
    \kwret \; \expr \mid
    \kwif \; \expr \; \lab\\

    \text{References} & \refer &::=&
    x \mid
    \expr [ \expr ]\\

    \text{Expressions} & \expr &::=&
    \pval \mid
    \lambda x. \; \lab \mid
    \refer \mid
    \op(\expr^*)\\
  \end{array}
\]

A program $\prog$ is a sequence of labeled instructions. An instruction $\inst$
is an expression assignment, an object creation, a function call, a return
instruction, or a branch.  A reference $\refer$ is a variable or a property
access of an object.  An expression $\expr$ is a primitive, a lambda function, a
reference, or an operation between other expressions.

\[
  \begin{array}{lr@{~}c@{~}l@{~}c@{~}l}
    \text{States} & \st &\in& \stset &=& \labset \times \memset \times
    \ctxtset \times \eaddrset\\
    \text{Memories} & \mem &\in& \memset &=& \locset \finmap \valset\\
    \text{Contexts} & \ctxt &\in& \ctxtset &=& \eaddrset \finmap (\eaddrset
    \times \labset \times \locset)\\
    \text{Locations} & \loc &\in& \locset &=& (\eaddrset \times \varset) \uplus
    (\oaddrset \times \strset)\\
    \text{Values} & \val &\in& \valset &=& \pvalset \uplus \oaddrset \uplus
    \fvalset\\
    \text{Primitives} & \pval &\in& \pvalset &=& \strset \uplus \cdots\\
    \text{Addresses} & \addr &\in& \addrset &=& \eaddrset \uplus \oaddrset\\
    \text{Functions} & \fval{x}{\lab} &\in& \fvalset &=& \varset \times
    \labset\\
  \end{array}
\]

States $\stset$ consist of labels $\labset$, memories $\memset$, contexts
$\ctxtset$, and environment addresses $\eaddrset$.  A memory $\mem \in \memset$
is a finite mapping from locations to values.  A context $\ctxt \in \ctxtset$ is
a finite mapping from environment addresses to tuple of environment addresses,
return labels, and left-hand side locations.  A location $\loc \in \locset$ is a
variable or an object property; a variable location consists of an environment
address and its name, and an object property location consists of an object
address and a string value.  A value $\val \in \valset$ is a primitive, an
address, or a function value.  An address $\addr \in \addrset$ is an environment
address or an object address.  A function value $\fval{x}{\lab} \in \fvalset$
consists of a parameter name and a body label.  In the core language, the closed
scoping is used for functions for brevity, thus only parameters and local
variables are accessible in a function body.

We formulate the concrete semantics of the core language.  The transition
relation between concrete states is defined with the semantics of references and
expressions using two different forms \fbox{$\referrule{\st}{\refer}{\loc}$} and
\fbox{$\exprrule{\st}{\expr}{\val}$}, respectively.  The initial states are
$\istset = \{ (\ilab, \varnothing, \epsilon, \tladdr) \}$ where $\ilab$ denotes
the initial label, $\epsilon$ empty map, and $\tladdr$ the top-level environment address.

\subsection{Abstract Semantics}
In the abstract semantics of the core language, we use the flow sensitivity with a
flow sensitive view abstraction $\fsviewmap: \labset \rightarrow \dom$ that
discriminates states using their labels: $\forall \lab \in \labset. \;
\fsviewmap(\lab) = \{ \st \in \stset \mid \st = (\lab, \_, \_, \_) \}$. Thus, the
sensitive abstract domain is defined as $\sabsdom = \labset \rightarrow
\absdom$.  We define an abstract state $\abselem \in \absdom$ as a tuple of an
abstract memory, an abstract context, an abstract address, and an
abstract counter as follows:

\[
  \begin{array}{l@{~}r@{~}c@{~}l@{~}c@{~}l}
\text{Abstract states} & \abselem &\in& \absdom &=& \absmemset \times \absctxtset
\times \absaddrset \times \abscountset\\
\text{Abstract memories} & \absmem &\in& \absmemset &=& \abslocset \finmap
\absvalset\\
\text{Abstract locations} & \absloc &\in& \abslocset &=& (\absaddrset \times
\varset) \uplus (\absaddrset \times \strset)\\
\text{Abstract addresses} & \absaddr &\in& \absaddrset &=& \labset\\
\text{Abstract contexts} & \absctxt &\in& \absctxtset &=& \absaddrset \finmap
\powerset{\absaddrset \times \viewset \times \powerset{\abslocset}}\\
\text{Abstract counters} & \abscount &\in& \abscountset &=& \absaddrset
\rightarrow \{ \abszero, \absone, \absmany \}\\
\text{Abstract values} & \absval &\in& \absvalset &=& \powerset{\pvalset
\uplus \absaddrset \uplus \fvalset}\\
  \end{array}
\]

An abstract memory $\absmem \in \absmemset$ is a finite mapping from abstract
locations $\abslocset$ to abstract values $\absvalset$.  Abstract locations
$\abslocset$ are pairs of abstract addresses with variable names or string
values. Abstract addresses $\absaddrset$ are defined with the
\textit{allocation-site abstraction} that partitions concrete addresses
$\addrset$ based on their allocation sites $\labset$.  Abstract contexts
$\absctxtset$ are finite maps from abstract addresses to powersets of triples of
abstract addresses, views, and powerset of abstract locations.  For abstract
counting~\cite{abstract-gc-counting, revisit-recency} in static analysis, we
define abstract counters $\abscountset$ that are mappings from abstract addresses to
their abstract counts representing how many times each abstract address has been
allocated; $\abszero$ denotes that it has never been allocated, $\absone$ once,
and $\absmany$ more than or equal to twice.

We define the semantics of the view transition for the core language.  For abstract
memories, we use the notation $\absmem[L \mapstos \absval]$ to represent the
update of multiple abstract locations in $L$ with the abstract value $\absval$.
It performs the strong update if the abstract address for an abstract location
$(\absaddr, \_) \in L$ is singleton: $\abscount(\absaddr) = \absone$.
Otherwise, it performs the weak update for the analysis soundness.  We use
the increment function $\inc: \abscountset \times \absaddrset \rightarrow
\abscountset$ of the abstract counter defined as follows:
\[
  \inc(\abscount)(\absaddr_0) = \lambda \absaddr \in \absaddrset. \; \left\{
    \begin{array}{ll}
      \absone & \text{if} \; \absaddr = \absaddr_0 \wedge
      \abscount(\absaddr_0) = \abszero\\
      \absmany & \text{if} \; \absaddr = \absaddr_0 \wedge
      \abscount(\absaddr_0) = \absone\\
      \abscount(\absaddr) & \text{otherwise}
    \end{array}
  \right.
\]


\subsection{{\SealeD} Execution}

We define the {\sealed} states by not only extending the concrete values
$\valset$ with {\sealed} values $\symbset$ but also adding the abstract counters
$\abscountset$ as follows:
\[
  \begin{array}{r@{~}c@{~}l}
    \symbstset &=& \labset \times \memset \times \ctxtset \times \eaddrset
    \times \abscountset\\
    \ctxtset &=& \eaddrset \finmap ((\eaddrset \times \labset \times \locset)
    \uplus \symbset)\\
    \valset &=& \pvalset \uplus \oaddrset \uplus \fvalset \uplus \symbset\\
    \abscountset &=& \oaddrset \rightarrow \{ \abszero, \absone, \absmany \}\\
  \end{array}
\]
Because JavaScript provides open objects, the properties of objects can be dynamically added or deleted.
Moreover, since object properties are string values that can be constructed at run time,
it is difficult to perform sound strong updates in static analysis.
To check the possibility of strong updates during {\sealed} execution,
we augment its states with the abstract counters $\abscountset$.

For each abstract value in a given abstract state,
if the abstract value denotes a single concrete value,
the converter $\asconverter: (\viewset \times
\absdom) \rightarrow (\absimapset \times \symbstset)$
keeps it; otherwise, $\asconverter$ replaces the abstract
value with its unique identifier and maintains the mapping from the
unique identifier to the abstract value to construct an abstract instantiation map.
The opposite converter $\saconverter: (\absimapset \times
  \symbstset)  \rightarrow (\viewset \times \absdom)$
recovers abstract values from their unique identifiers using the abstract instantiation map.
We define the {\sealed} transition relation $\symbtrans$
only if the next step does not require actual values of any {\sealed} values.
Otherwise, a given {\sealed} state does not have any {\sealed}
transitions to apply.  For example, we add the following rule for the $\kwret$ statement:
\begin{mathpar}
  \inferrule
  {
    \prog(\lab) = \kwret \; \expr\\
    \exprrule{\symbst}{\expr}{\val}\\
    \ctxt(\addr) \in \symbset
  }
  {
    \symbst = (\lab, \mem, \ctxt, \addr, \abscount)\
    \symbtrans \excst
  }
\end{mathpar}
We extend each rule of the concrete semantics to support such behaviors of {\sealed} 
values.

\section{Implementation}\label{sec:implementation}
We implemented JavaScript static analysis using dynamic shortcuts
presented in Section~\ref{sec:javascript} in a prototype implementation dubbed
$\tool$.  The tool is an extension of an existing state-of-the-art JavaScript
static analyzer SAFE~\cite{safe, safe2} with a dynamic analyzer
Jalangi~\cite{jalangi}, and it is an open-source project and available online~\footnote{The
URL of the tool is anonymized due to a double-blind review process.}.  In this
section, we introduce challenges and solutions in implementing dynamic
shortcuts on existing JavaScript analyzers.

\paragraph{Sealed Values.}
The main challenge of implementing dynamic shortcuts is to support sealed
symbolic execution on an existing JavaScript engine.  To represent an abstract
value, we use the \jscode{Proxy} object introduced in ECMAScript 6
(2015, ES6)~\cite{es6}, which allows developers to handle internal behaviors
of specific objects such as property reads and writes and implicit conversions.
We are inspired by \textsc{Mimic}~\cite{mimic}, which used \jscode{Proxy} to
capture accesses from internals of opaque functions.  When the dynamic analyzer
constructs an execution environment at the start of a dynamic shortcut, it
creates \jscode{Proxy} objects to represent abstract values via the
following \jscode{generateSymbol} function:
\begin{lstlisting}[style=myJSstyle]
function generateSymbol() {
  function detect() { /* access detection */ }
  return new Proxy(function() {}, {
    getPrototypeOf: detect,  ...
    construct     : detect
  }); }
var x = generateSymbol();
var y = x;
var z = x + 1;
\end{lstlisting}
The function creates a sealed symbol as a proxy object with a dummy
function object and a handler for all 13 traps using an access detection
function \jscode{detect}.  A sealed symbol invokes the function \jscode{detect}
when any of 13 pre-defined traps are operated on the object, which enables us to
determine whether an object is sealed or not.  For example, the variable
\jscode{y} successfully points to the same symbol stored in \jscode{x}, but the
program invokes the function \jscode{detect} on line 9 because \jscode{x + 1} requires
the actual value of the symbol.  In addition, we instrument unary and binary
operations in Jalangi so that we can detect all the accesses on the
symbol beyond the 13 traps provided by \jscode{Proxy}.
Using this idea, we successfully extended the
JavaScript engine to support sealed symbolic execution.

\paragraph{Synchronization of Control Points.}
For seamless interaction between static analysis and sealed symbolic execution,
synchronization of control points in both worlds is necessary.
The SAFE static analyzer parses a JavaScript program to an Abstract
Syntax Tree (AST), compiles it to its intermediate representation, and
builds a Control Flow Graph (CFG).  It produces annotations for
control points on CFGs including functions, call sites, and object allocation sites.
On the contrary, the Jalangi dynamic analyzer instruments a JavaScript program 
to keep track of necessary information at run time.  Therefore, during
sealed symbolic execution, Jalangi requires the annotations from SAFE to compute
abstract locations of newly created concrete objects and lexical environments.
The key to put annotations on the instrumented code is the source-code
location of the original AST node.  While both CFG of SAFE and
instrumented code of Jalangi maintain the source-code location information,
because they use different parsers we found various location mismatches for corner cases.
Therefore, we synchronize control points of two analyzers by using the closest match
of their source-code locations rather than using their exact match.

\paragraph{Function-Level Dynamic Shortcut.}
The dynamic shortcut is activated when the current abstract state passes the
filter $\checker$.  If the filter tolerantly admits the dynamic shortcut, the
analysis may suffer from the frequent communication between static and dynamic
analyzers.  To adjust the burden of communication, we only activate new
dynamic shortcut in function entries and deactivate it in the corresponding
function exits thus $\tool$ supports only \textit{function-level} dynamic
shortcut.

\paragraph{Termination.}
To guarantee the termination of static analysis using dynamic shortcuts, the
converter $\asconverter$ should pass an analysis element $(\view, \abselem)$
only when it terminates in a time bound $N$.  Since statically checking the
termination property is a difficult task, we simply perform sealed symbolic
execution with a pre-determined time limit.  When it times out,
we treat it as a failure in conversion;
otherwise, we use the result of sealed symbolic execution.  Our experiments used
5 seconds as the time limit for each sealed symbolic execution.


% \paragraph{Communication between Analyzers.}
% To perform dynamic shortcut, the static analyzer should communicate with the the
% dynamic analyzer by passing the current abstract state and receiving the final
% result of the sealed symbolic execution.  However, The static analyzer SAFE and
% the dynamic analyzer Jalangi are implemented in different languages Scala and
% JavaScript, respectively.  To overcome the difference, we represent abstract
% states as JSON objects and communicate between analyzers by passing them.  For
% each dynamic shortcut, Jalangi constructs the execution environments based on
% the given JSON object and executes the dynamic analysis from the target program
% points.  After finishing the sealed symbolic execution, it sends a newly
% constructed JSON object to SAFE to update the abstract states in the function
% exit point.


%\paragraph{Abstract Locations for Objects.}
%During dynamic analysis, each JavaScript object should be designated by the
%corresponding abstract location used in the static analysis.  In this paper, we
%define abstract locations using the allocation site
%abstraction~\cite{allocation-site} with heap cloning~\cite{heap-cloning} for
%each abstract context in the static analyzer.  Thus, we instrument the given
%JavaScript program to annotate each object created during dynamic analysis with
%its allocation site and context information.  After the sealed symbolic
%execution, the dynamic analyzer collects the objects based on their annotated
%information for each abstract location.



\section{Evaluation}\label{sec:eval}

We developed $\tool$ to implement dynamic shortcut technique for JavaScript
static analysis (Section~\ref{sec:javascript}).  Our tool is an extension of
a state-of-the-art JavaScript static analyzer SAFE 2.0~\cite{safe2} with the
dynamic analyzer Jalangi 2~\cite{jalangi}.

\todo - Explain the detail of implementation

We evaluated our tool based on the following research questions:
\begin{itemize}
  \item \textbf{RQ1) Analysis Speed-up:} How much analysis time is reduced by
    adding dynamic shortcut to static analysis?
  \item \textbf{RQ2) Precision Improvement:} How much analysis precision is
    improved by replacing the manual modeling with dynamic shortcut?
  \item \textbf{RQ3) Opaque Function Coverage:} How many opaque functions are
    covered by dynamic shortcut without using manual modeling?
\end{itemize}
We performed our experiments on an Ubuntu machine equipped with 4.2GHz Quad-Core
Intel Core i7 and 64GB of RAM.


\subsection{Analysis Speed-up}

\begin{table}
  \caption{Analysis of 306 tests of Lodash 4.}
  \label{table:conc-test}
  \vspace*{-1em}
  \centering
  \[
    \begin{array}{c?r|r|r}
      &
      \multicolumn{1}{c|}{\text{SAFE (ms)}} &
      \multicolumn{1}{c|}{\tool \text{ (ms)}} &
      \multicolumn{1}{c}{\text{Speed Up}}\\\hline\hline
      \text{avg.}       & \inred{168,399} & \inred{1,811} & \inred{116.38}\\\hline
      \text{min.}       & \inred{ 37,095} & \inred{  514} & \inred{ 12.81}\\\hline
      \text{max.}       & \inred{299,375} & \inred{2,986} & \inred{489.67}\\\hline\hline
      \text{\# success} & \inred{    274} & \inred{  306} &
    \end{array}
  \]
  \vspace*{-1em}
\end{table}

We targeted official 306 tests of Lodash 4\cite{lodash} used in motivating
examples (Section~\ref{sec:motivation}).  Recent papers for JavaScript static
analysis techniques~\cite{value-refinement, value-partitioning} also evaluated
their techniques based on them.  Table~\ref{table:conc-test} shows that our tool
successfully analyzes \inred{all 306} tests in \inred{514 ms} on average while
SAFE only analyzes \inred{274} out of 306 tests within 5 minute timeout.  For
\inred{274} tests analyzable by SAFE, our tool is at least \inred{12.81x}, at
most \inred{489.67x}, and \inred{116.38x} on average faster than SAFE.

However, all tests do not contain any non-deterministic values thus they could
be analyzed via only dynamic analysis without static analysis.  In fact, our
tool covers all program parts via only one dynamic shortcut without using
abstract semantics of static analysis.  Thus, we modified 306 official tests to
include abstract values by lifting each literal to the corresponding typed
value.  For example, the numerical literal \jscode{42} will be lifted to the
abstract value $\top_{\code{num}}$, which represents the whole numerical values.

% abstract versions
% of official 
% 
% performs only dynamic shortcut
% 
% by only performing dynamic shortcut for
% whole program points.  Moreover, they analyzed 
% 
% , which is a modern JavaScript library
% delivering modularity, performance, and extras.  It is the first 
% 
% 
% For motivating examples, we excerpt the \jscode{concat} function in
% Figure~\ref{fig:concat} from Lodash library~\cite{lodash} (v4.17.20), which is
% the most popular npm package\footnote{https://www.npmjs.com/browse/depended}
% and \inred{124,562} npm packages have dependency with it.  The \jscode{concat}
% 

\subsection{Opaque Function Coverage}
To evaluate how much manual modeling efforts of opaque functions
are reduced by dynamic shortcuts, we measured the number of tests where
opaque functions that are analyzed only by dynamic analysis not by
static analysis.  Table~\ref{table:func-replace} summarizes the result.
For 198 original tests and 143 abstracted tests, we measured the
number of tests that use only dynamic shortcuts instead of manual modeling
for each JavaScript built-in library function.  For each row,
\textbf{Object} column denotes a built-in object, \textbf{Function} a function
name, and \textbf{\# Replaced} the number of tests successfully replacing manual
modeling via dynamic shortcuts with the total number of tests using the target function.
For example, the first row in the left side describes that \jscode{Array} is used in
203 original tests and 116 abstracted tests.  Among them, 203 original
tests 90 abstracted tests are successfully analyzed by dynamic shortcuts
instead of using the modeling of \jscode{Array}.  Each filled cell describes
fully replaceable cases in Table~\ref{table:func-replace}.  Therefore, dynamic
shortcuts effectively lessen the burden of manual modeling for JavaScript
built-in functions.  Again, since all the original Lodash 4 tests except one
test is finisehd with a single dynamic shortcut, all the built-in fuctions
except \jscode{Math.floor} and \jscode{Function.prototype.call} are replaceable
for original tests.  For abstracted tests, 12 functions out of 60 built-in
functions are analyzed by only dynamic shortcut instead of manual modeling.

\section{Rights Information}

Authors of any work published by ACM will need to complete a rights
form. Depending on the kind of work, and the rights management choice
made by the author, this may be copyright transfer, permission,
license, or an OA (open access) agreement.

Regardless of the rights management choice, the author will receive a
copy of the completed rights form once it has been submitted. This
form contains \LaTeX\ commands that must be copied into the source
document. When the document source is compiled, these commands and
their parameters add formatted text to several areas of the final
document:
\begin{itemize}
\item the ``ACM Reference Format'' text on the first page.
\item the ``rights management'' text on the first page.
\item the conference information in the page header(s).
\end{itemize}

Rights information is unique to the work; if you are preparing several
works for an event, make sure to use the correct set of commands with
each of the works.

The ACM Reference Format text is required for all articles over one
page in length, and is optional for one-page articles (abstracts).



\section{Conclusion}\label{sec:conclusion}
We presented a novel technique for JavaScript static analysis using \textit{dynamic shortcuts}.
It can significantly accelerate static
analysis by freely leveraging high performance of dynamic analysis for
concretely executable program parts.  To maximize such benefits,
we proposed \textit{sealed symbolic execution}, which performs
concrete execution using sealed symbolic values for abstract values.
We formally defined static analysis using dynamic shortcuts in the
abstract interpretation framework and proved its soundness and termination.
We developed $\tool$ as a prototype implementation of the proposed approach
by extending a combination of the state-of-the-art static and dynamic
analyzers SAFE and Jalangi.  Our tool accelerates the speed
of static analysis 19.96$\x$ for original tests and 6.30$\x$ for
abstracted tests of Lodash 4 library.  Moreover, it detects 6 more
dead branches by using sealed symbolic execution instead of
manual modeling for 12 opaque functions on average.


%%
%% The acknowledgments section is defined using the "acks" environment
%% (and NOT an unnumbered section). This ensures the proper
%% identification of the section in the article metadata, and the
%% consistent spelling of the heading.
%\begin{acks}
%To Robert, for the bagels and explaining CMYK and color spaces.
%\end{acks}

%%
%% The next two lines define the bibliography style to be used, and
%% the bibliography file.
\bibliographystyle{ACM-Reference-Format}
\bibliography{ref}

%%
%% If your work has an appendix, this is the place to put it.
%\appendix

\end{document}
\endinput
%%
%% End of file `sample-sigconf.tex'.
